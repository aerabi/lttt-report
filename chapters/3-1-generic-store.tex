\chapter{Generic Store}\label{chap:generic-store}

\newcommand{\angkv}{\langle k, v \rangle}
\newcommand{\angkvp}{\langle k', v' \rangle}

We are dealing with linear type system, in which exchange is allowed. Hence for convenience, we can add commutativity to the data structure that we use for our stores.


\begin{definition}
The generic stores are denoted by $S$ in this chapter, which are unordered lists of keys $k$ and values $v$, defined inductively as follows:
\begin{equation*}
S ::= \emptyset~|~S :: \angkv.
\end{equation*}
The empty store is denoted by $\emptyset$ and $::$ is the append operator.
\end{definition}

\begin{definition}
Concatenation of two stores are defined as follows:

\begin{equation*}
S \circ S' =
\begin{cases}
    S, & S' = \emptyset, \\
    (S \circ S'') :: \angkv, & S' = S'' :: \angkv.
\end{cases}
\end{equation*}
\end{definition}

\begin{lemma}[Associativity]
For stores $S_1, S_2, S_3$, we have $(S_1 \circ S_2) \circ S_3 = S_1 \circ (S_2 \circ S_3)$.
\end{lemma}
\begin{proof}
By induction on $S_2$.
\end{proof}

\begin{definition}[Membership]
\label{defn:generic-context-membership}
By definition, stores are lists of ordered pairs. Membership is defined recursively as follows:
\begin{itemize}
    \item If $S = S' :: \angkv$, then $\angkv \in S$,
    \item If $\angkvp \in S'$ and $k \not= k'$, then $\angkvp \in S :: \angkv$.
\end{itemize}
\end{definition}

The membership is defined so, so that appending a pair with the same key and a different value would overwrite the old record, in a sense:
\begin{itemize}
    \item $\angkv \in \emptyset :: \angkv$,
    \item $\angkv \not\in \emptyset :: \angkv :: \ang{k, v'}$.
\end{itemize}

\begin{definition}
A key $k$ is said to be a key in the store $S$, denoted as $\ang{k, \cdot} \in S$, iff for some value $v$ we have $\angkv \in S$.
\end{definition}

\begin{lemma}
Let $B$ and $C$ be stores, $k$ a key, and $v$ a value. Then:
\begin{itemize}
    \item If $\angkv \in C$, then $\angkv \in B \circ C$,
    \item If $\angkv \in B$ and $\ang{k, \cdot} \not\in C$, then $\angkv \in B \circ C$,
    \item If $\angkv \in B \circ C$, then either $\angkv \in B$ or $\angkv \in C$.
\end{itemize}
\end{lemma}

\begin{definition}[Equivalence]
We say the store $S$ is a subset of the store $S'$, writing $S \sqsubseteq S'$, iff for all $\angkv \in S$ we have $\angkv \in S'$. We say the two stores are equivalent, writing $S \equiv S'$, when $S \sqsubseteq S'$ and $S' \sqsubseteq S$.
\end{definition}

\begin{lemma}
The subset relation $\sqsubseteq$ on the stores is reflexive and transitive. Furthermore, the equivalence relation $\equiv$ is reflexive, transitive, and symmetric.
\end{lemma}

\begin{definition}[Normality]
We say a store $S$ is duplicated, if it has duplicated keys, recursively defined as follows:
\begin{itemize}
    \item If $S = S' :: \angkv$ and $\ang{k, \cdot} \in S'$, then $S$ is duplicated,
    \item If $S = S' :: \angkv$ and $S'$ is duplicated, so is $S$.
\end{itemize}
If $S$ is not duplicated, we say it's normal.
\end{definition}

\begin{theorem}
Let $B$ and $C$ be stores. If $B \circ C$ is normal, then $B \circ C \equiv C \circ B$.
\end{theorem}