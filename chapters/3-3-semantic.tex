\section{Operational Semantics}

\newcommand{\dbrack}[1]{\left\llbracket #1 \right\rrbracket}

\begin{figure}
    \centering
    \begin{eqnarray*}
\pi_i(\v_1, \v_2) &\leadsto& \v_i\\
(\lambda x.\t_1)\v_2 &\leadsto& \t_1 [x \mapsto \v_2]\\
\mathtt{case}\ \ \mathtt{in}_i\v \ \ \mathtt{of}\ (\mathtt{in}_1\ x_1 \mapsto \t_1\  \lvert\  \mathtt{in}_2 \ x_2 \ \mapsto \t_2) &\leadsto& \t_i [x_i \mapsto \v]\\
\mathtt{let}\ ()=()\ \mathtt{in} \ \e &\leadsto& \e\\
\mathtt{let}\ (x_1,x_2)=(\e_1,\e_2)\ \mathtt{in}\ \e &\leadsto& \e[x_1 \mapsto \e_1, x_2 \mapsto \e_2]\\
(\lambda x.\e_1)\e_2 &\leadsto& \e_1[x \mapsto \e_2]\\
\mathtt{bind}\ x = \mathtt{return} \ \e_1\ \mathtt{in} \ \e_2\ &\leadsto&\ \e_2 [x \mapsto \e_1]\\
\mathtt{force}( \mathtt{suspend}\ \e)\ &\leadsto& \e\\
\mathtt{let}\  \floor{x}=\floor{\v}\  \mathtt{in}\ \e\ &\leadsto& \e[x\mapsto \v]\\
\mathtt{case}\ \ \mathtt{in}_i\e \ \ \mathtt{of}\ (\mathtt{in}_1\ x_1 \mapsto \e_1\  \lvert\  \mathtt{in}_2 \ x_2 \mapsto \e_2) &\leadsto& \e_i [x_i \mapsto \e]
\end{eqnarray*}
    \caption{Operational Semantics Step Relation}
    \label{fig:operational-semantics-step}
\end{figure}

In order to approach the semantics, a normalized version of the terms and expression is defined:
\begin{align*}
    \v ::=&~ () \mid (\v_1, \v_2) \mid \mathtt{in}_i \v \mid \lambda x. \t \mid \mathtt{suspend}~\e \\
    \n ::=&~ () \mid (\e_1, \e_2) \mid \mathtt{in}_i \e \mid \lambda x. \e \mid \mathtt{return}~\e \mid \floor{\v}
\end{align*}
Here, $\e$ and $\n$ are normalized versions of $\t$ and $\e$, respectively. A step relation $\leadsto$ is defined on $\t \cup \e$ in \figref{fig:operational-semantics-step}. Using the step relation, a full operational semantics using context hole is defined in Figures \ref{fig:small-step-operational-semantics} and \ref{fig:operational-semantics-rules}.

\begin{figure}
    \centering
    \begin{align*}
    \mathfrak{E}_{\t} ::=&~ ([~], \t) \mid (\v, [~]) \mid \pi_i [~] \mid \mathtt{in}_i [~] \\
    & \mid \mathtt{case}~[~]~\mathtt{of}~(\mathtt{in}_1 x_1 \rightarrow \t_1 \mid \mathtt{in}_2 x_2 \rightarrow \t_2) \\
    & \mid [~] \t \mid \v [~] \mid \mathtt{force}~[~] \mid \floor{[~]} \\
    \mathfrak{E}_\e ::=&~ \mathtt{let}~() = [~]~\mathtt{in}~\e \mid \mathtt{let}~(x_1, x_2) = [~]~\mathtt{in}~\e \\
    & \mid \mathtt{case}~[~]~\mathtt{of}~(\mathtt{in}_1 x_1 \rightarrow \e_1 \mid \mathtt{in}_2 x_2 \rightarrow \e_2) \\
    & \mid [~] \e \mid \mathtt{bind}~x = [~]~\mathtt{in}~\e \mid \mathtt{let}~\floor{x} = [~]~\mathtt{in}~\e
\end{align*}

    \caption{Operational Semantics Evaluation Context}
    \label{fig:small-step-operational-semantics}
\end{figure}

\begin{figure}
    \centering
    \[
    \infer{\mathfrak{E}_\t \dbrack{\t} \leadsto \mathfrak{E}_\t \dbrack{\t'}}{\t \leadsto \t'}
    ~~~~ ~~~~
    \infer{\mathfrak{E}_\e \dbrack{\e} \leadsto \mathfrak{E}_\e \dbrack{\e'}}{\e \leadsto \e'}
    \]
    \caption{Operational Semantics Reduction Rules}
    \label{fig:operational-semantics-rules}
\end{figure}

\begin{theorem}[Preservation]
\label{theorem:preservation}
If $\vdash \t : \T$ and $\t \leadsto \t'$ then $\vdash \t' : \T$. If $\vdash \e : \A$ and $\e \leadsto \e'$ then $\vdash \e' : \A$.
\end{theorem}

Theorem \ref{theorem:preservation} is stated in work of \cite{Paykin2016TheEO}. Stronger results are proved in the next chapter using Coq. Another unproved theorem stated in work of Paykin is the following:

\begin{theorem}[Progress]
\label{theorem:progress}
If $\vdash \t : \T$ then either $\t$ is a value or $\t$ can take a step. If $\vdash \e : \A$ then either $\e$ is in weak head normal form or $\e$ can take a step.
\end{theorem}