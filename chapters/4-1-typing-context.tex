\chapter{Generic Store}\label{chap:generic-store}

\newcommand{\angkv}{\langle k, v \rangle}
\newcommand{\angkvp}{\langle k', v' \rangle}

Typing environments lie at the core of the type inference relations, storing what variables have which types at the moment. The rule
\[
\Gamma :: \ang{x, T} \vdash x : T
\]
is a good example: A variable $x$ is typed to $T$, if it was declared to be of type $T$, hence stored in the typing environment.

When implementing type systems in Coq, choosing a right data model for the typing environments is very important, as it will play a big role is the proofs about the typing rules. Therefore, different attempts were made to create general data models for typing environments.

Among his libraries for the locally nameless infrastructure, Arthur \cite{DBLP:journals/jar/Chargueraud12} built a generic model for environments, and later Emmanuel \cite{DBLP:journals/corr/abs-1112-1316} rewrote his generic environments model in Coq 8.3, publishing it as a standalone Coq library\footnote{\url{https://github.com/coq-community/generic-environments}}.

\todo{Add details about problems of these libraries}


Current chapter focuses on the new implementation of the typing environments in Coq.

\section{A New Implementation}

As the data model can hold any type of data, and not only typing information, we call it \textit{generic store} or simply \textit{store}, when merely talking about the abstract data structure.

\begin{definition}
The generic stores are denoted by $S$ in this chapter, which are unordered lists of keys $k$ and values $v$, defined inductively as follows:
\begin{equation*}
S ::= \emptyset~|~S :: \angkv.
\end{equation*}
The empty store is denoted by $\emptyset$ and $::$ is the append operator.
\end{definition}

We would need decidablity of equality on the keys and values of the store. So, we assume the keys are chosen from a class $\mathbf{K}$ equipped with a reflexive relation $\asymp$ with the following property:
\begin{itemize}
    \item If $k_1 \asymp k_2$, then $k_1 = k_2$.
\end{itemize}
The axiom of extentionality is not required for the class of values, $\mathbf{V}$. See \listref{listing:kvmoduletype} for implementation details.

\begin{listing}[t]
\begin{minted}[
frame=lines,
framesep=2mm,
baselinestretch=1.2,
fontsize=\footnotesize,
linenos
]{coq}
Module Type KeyModuleType.

  Parameter T : Type.
  Parameter equal : T -> T -> bool.
  Parameter eq_refl : forall x, equal x x = true.
  Parameter eq_extensionality : forall x y, equal x y = true -> x = y.

End KeyModuleType.

Module Type ValModuleType.

  Parameter T : Type.
  Parameter equal : T -> T -> bool.
  Parameter eq_refl : forall x, equal x x = true.

End ValModuleType.
\end{minted}
\caption{Definition of the classes $\mathbf{K}$ and $\mathbf{V}$ in Coq}
\label{listing:kvmoduletype}
\end{listing}

\begin{definition}
Concatenation of two stores are defined as follows:

\begin{equation*}
S \circ S' =
\begin{cases}
    S, & S' = \emptyset, \\
    (S \circ S'') :: \angkv, & S' = S'' :: \angkv.
\end{cases}
\end{equation*}
\end{definition}

\begin{lemma}[Associativity]
For stores $S_1, S_2, S_3$, we have $(S_1 \circ S_2) \circ S_3 = S_1 \circ (S_2 \circ S_3)$.
\end{lemma}
\begin{proof}
By induction on $S_2$.
\end{proof}

\begin{definition}[Membership]
\label{defn:generic-context-membership}
By definition, stores are lists of ordered pairs. Membership is defined recursively as follows:
\begin{itemize}
    \item If $S = S' :: \angkv$, then $\angkv \in S$,
    \item If $\angkvp \in S'$ and $k \not= k'$, then $\angkvp \in S :: \angkv$.
\end{itemize}
\end{definition}

The membership is defined so, so that appending a pair with the same key and a different value would overwrite the old record, in a sense:
\begin{itemize}
    \item $\angkv \in \emptyset :: \angkv$,
    \item $\angkv \not\in \emptyset :: \angkv :: \ang{k, v'}$.
\end{itemize}

\begin{definition}
A key $k$ is said to be a key in the store $S$, denoted as $\ang{k, \cdot} \in S$, iff for some value $v$ we have $\angkv \in S$.
\end{definition}

\begin{lemma}
Let $B$ and $C$ be stores, $k$ a key, and $v$ a value. Then:
\begin{itemize}
    \item If $\angkv \in C$, then $\angkv \in B \circ C$,
    \item If $\angkv \in B$ and $\ang{k, \cdot} \not\in C$, then $\angkv \in B \circ C$,
    \item If $\angkv \in B \circ C$, then either $\angkv \in B$ or $\angkv \in C$.
\end{itemize}
\end{lemma}

\begin{definition}[Equivalence]
We say the store $S$ is a subset of the store $S'$, writing $S \sqsubseteq S'$, iff for all $\angkv \in S$ we have $\angkv \in S'$. We say the two stores are equivalent, writing $S \equiv S'$, when $S \sqsubseteq S'$ and $S' \sqsubseteq S$.
\end{definition}

\begin{lemma}
The subset relation $\sqsubseteq$ on the stores is reflexive and transitive. Furthermore, the equivalence relation $\equiv$ is reflexive, transitive, and symmetric.
\end{lemma}

\begin{definition}[Normality]
We say a store $S$ is duplicated, if it has duplicated keys, recursively defined as follows:
\begin{itemize}
    \item If $S = S' :: \angkv$ and $\ang{k, \cdot} \in S'$, then $S$ is duplicated,
    \item If $S = S' :: \angkv$ and $S'$ is duplicated, so is $S$.
\end{itemize}
If $S$ is not duplicated, we say it's normal.
\end{definition}

\begin{theorem}
Let $B$ and $C$ be stores. If $B \circ C$ is normal, then $B \circ C \equiv C \circ B$.
\end{theorem}