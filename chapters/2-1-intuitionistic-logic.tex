\section{Intuitionistic Logic}

The \textit{intuitionistic logic} is known to differ from the classical logic by rejecting the \textit{law of excluded middle} (Latin: \textit{principium tertii exclusi}, aka \textit{tertium non datur}). As a result, the two following classical tautologies are not valid in the intuitionistic logic: $p \vee \neg p$ (\textit{tertium non datur}) and $\neg \neg p \rightarrow p$ (double-negation elimination).

\subsection{Language of Propositional Intuitionistic Logic}

The language of intuitionistic propositional logic, IPL, consists of the following:

\begin{itemize}
    \item Variables: $x$,
    \item Function symbols: $f, g, C, \dots$,
    \item Predicate symbols: $P, Q, p, q, \dots$,
    \item Logical symbols: $\bot, \top, \rightarrow, \vee, \wedge, \neg$.
\end{itemize}

Function symbols can be of any arity. If the arity is 0, we call it a constant, and show it with a capital letter like $C$. If the arity is 1, we usually omit the parentheses when applying it: $fx$. When the arity is 2, we show is as an infix: $x_1 + x_2$.

These are also the case for the predicates. A predicate of arity 0 is shown with a small Latin letter $p$ or $q$ and is called an atom. A special atom is $\bot$ which is called falsity or contradiction.

\begin{definition}[Term]
Every variable in IPL is a term. Furthermore, if $t_1, \dots, t_n$ are terms and $f$ is function symbol of arity $n$, then $f(t_1, t_1, \dots, t_{n})$ is also a term.
\end{definition}

\begin{definition}[Atomic Formula]
If $t_1, \dots, t_n$ are terms and $P$ is predicate symbol of arity $n$, then $P(t_1, \dots, t_n)$ is called an \textit{atomic formula}.
\end{definition}

\begin{definition}[Formula]
Every atomic formula is a formula. Furthermore, if $\alpha_1$ and $\alpha_2$ are formulae, also are the following:
\[
\alpha_1 \rightarrow \alpha_2, \alpha_1 \vee \alpha_2, \alpha_1 \wedge \alpha_2.
\]
Here the connectives are: implication $\rightarrow$, disjunction $\vee$, and conjunction $\wedge$.
Negation of $\alpha$, $\neg\alpha$, is defined as $\alpha \rightarrow \bot$. Also, $\top$ is defined as $\neg\bot$.
\end{definition}

From now on, to introduce languages of logics, we will use the Backus--Naur form which is more economical (see \cite{DBLP:conf/ifip/Backus59}):
\begin{eqnarray*}
t &::=& x \mid f(t, \dots, t), \\
\alpha &::=& \bot \mid P(t, \dots, t) \mid \alpha \rightarrow \alpha \mid \alpha \vee \alpha \mid \alpha \wedge \alpha.
\end{eqnarray*}

As the logics we are going to study in this work have no terms, their languages are reduced to the formulae:
\[
\alpha ::= \bot \mid a \mid \alpha \rightarrow \alpha \mid \alpha \vee \alpha \mid \alpha \wedge \alpha,
\]
with $a$ ranging over atoms. To denote formulae, we usually use small Greek letters: $\alpha, \beta, \gamma, \varphi, \psi$. These symbols are variables in the metalanguage and do not belong to the language of IPL.

\subsection{Sequent and Natural Deduction}

A \textit{context} is a set of $\alpha$'s, usually denoted by $\Gamma$. We write $\Gamma, \Delta$ for $\Gamma \cup \Delta$, and also $\Gamma, \alpha$ for $\Gamma \cup \{\alpha\}$. 

\begin{definition}[Sequent]
The relation $\Gamma \vdash \alpha$ is defined syntactically by Figure \ref{fig:intuitionistic-logic-natural-deduction}, but can be though of as ``$\alpha$ holds if all $\Gamma$ holds''. The relation $\vdash$ is called \textit{sequent}, the elements of $\Gamma$ are called \textit{antecedents} or \textit{hypotheses}, and $\alpha$ is called \textit{succedents} or \textit{consequents}. When $\Gamma = \emptyset$, we write $\vdash \alpha$, and say that $\alpha$ is a \textit{theorem}.
\end{definition}

\begin{figure}
    \centering
    \begin{gather*}
\infer[\mbox{Ax}]{\Gamma, \alpha \vdash \alpha}{}
~~~~ ~~~~
\infer[\bot\mbox{-E}]{\Gamma \vdash \beta}{\Gamma \vdash \bot}
\\ \\ 
\infer[\wedge\mbox{-I}]{\Gamma \vdash \alpha_1 \wedge \alpha_2}{
    \Gamma \vdash \alpha_1
    &
    \Gamma \vdash \alpha_2
}
~~~~ ~~~~
\infer[\wedge\mbox{-E}]{\Gamma \vdash \alpha_i}{
    \Gamma \vdash \alpha_1 \wedge \alpha_2
}
\\ \\ 
\infer[\vee\mbox{-I}]{\Gamma\vdash \alpha_1 \vee \alpha_2}{\Gamma \vdash \alpha_i}
~~~~ ~~~~
\infer[\vee\mbox{-E}]{\Gamma \vdash \beta}{
    \Gamma \vdash \alpha_1 \vee \alpha_2
    &
    \Gamma, \alpha_1 \vdash \beta
    &
    \Gamma, \alpha_2 \vdash \beta
}
\\ \\ 
\infer[\rightarrow\mbox{-I}]{\Gamma \vdash \alpha \rightarrow \beta}{\Gamma, \alpha \vdash \beta}
~~~~ ~~~~
\infer[\rightarrow\mbox{-E}]{\Gamma \vdash \beta}{\Gamma \vdash \alpha & \Gamma \vdash \alpha \rightarrow \beta}
\end{gather*}
    \caption{Natural deduction rules of the propositional intuitionistic logic}
    \label{fig:intuitionistic-logic-natural-deduction}
\end{figure}

The proof system which consists of the sequent relation together with the deduction rules, is called \textit{natural deduction}.

The subset of IPL that only deals with $\alpha ::= x \mid \alpha \rightarrow \alpha$, with three rules of Ax, $\rightarrow$-I, and -E, is called IPL$(\rightarrow)$.