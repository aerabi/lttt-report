\section{Intuitionistic Logic}

The \textit{intuitionistic logic} is known to differ from the classical logic by rejecting the \textit{law of excluded middle} (Latin: \textit{principium tertii exclusi}, aka \textit{tertium non datur}). As a result, the two following classical tautologies are not valid in the intuitionistic logic: $p \vee \neg p$ (\textit{tertium non datur}) and $\neg \neg p \rightarrow p$ (double-negation elimination).

\subsection{Language of Propositional Intuitionistic Logic}

The language of intuitionistic propositional logic, IPC (C for calculus) or IPC$(\bot, \rightarrow, \vee, \wedge)$, is defined with the following grammar:
\[
\alpha ::= \bot \mid x \mid \alpha \rightarrow \alpha \mid \alpha \vee \alpha \mid \alpha \wedge \alpha.
\]
Here $\bot$ is constant (falsity or contradiction), $x$ ranges over the propositional variables, and the connectives are: implication $\rightarrow$, disjunction $\vee$, and conjunction $\wedge$. Negation $\neg \alpha$ is defined by $\neg \alpha = \alpha \rightarrow \bot$.

\subsection{Sequent and Natural Deduction}

A \textit{context} is a set of $\alpha$'s, usually denoted by $\Gamma$. We write $\Gamma, \Delta$ for $\Gamma \cup \Delta$, and also $\Gamma, \alpha$ for $\Gamma \cup \{\alpha\}$. 

\begin{definition}[Sequent]
The relation $\Gamma \vdash \alpha$ is defined syntactically by Figure \ref{fig:intuitionistic-logic-natural-deduction}, but can be though of as ``$\alpha$ holds if all $\Gamma$ holds''. The relation $\vdash$ is called \textit{sequent}, the elements of $\Gamma$ are called \textit{antecedents}, and $\alpha$ is called \textit{succedents} or \textit{consequents}. When $\Gamma = \emptyset$, we write $\vdash \alpha$, and say that $\alpha$ is a \textit{theorem}.
\end{definition}

\begin{figure}
    \centering
    \begin{gather*}
\infer[\mbox{Var}]{\Gamma, \alpha \vdash \alpha}{}
~~~~ ~~~~
\infer[\bot\mbox{-E}]{\Gamma \vdash \beta}{\Gamma \vdash \bot}
\\ \\ 
\infer[\wedge\mbox{-I}]{\Gamma \vdash \alpha_1 \wedge \alpha_2}{
    \Gamma \vdash \alpha_1
    &
    \Gamma \vdash \alpha_2
}
~~~~ ~~~~
\infer[\wedge\mbox{-E}]{\Gamma \vdash \alpha_i}{
    \Gamma \vdash \alpha_1 \wedge \alpha_2
}
\\ \\ 
\infer[\vee\mbox{-I}]{\Gamma\vdash \alpha_1 \vee \alpha_2}{\Gamma \vdash \alpha_i}
~~~~ ~~~~
\infer[\vee\mbox{-E}]{\Gamma \vdash \beta}{
    \Gamma \vdash \alpha_1 \vee \alpha_2
    &
    \Gamma, \alpha_1 \vdash \beta
    &
    \Gamma, \alpha_2 \vdash \beta
}
\\ \\ 
\infer[\rightarrow\mbox{-I}]{\Gamma \vdash \alpha \rightarrow \beta}{\Gamma, \alpha \vdash \beta}
~~~~ ~~~~
\infer[\rightarrow\mbox{-E}]{\Gamma \vdash \beta}{\Gamma \vdash \alpha & \Gamma \vdash \alpha \rightarrow \beta}
\end{gather*}
    \caption{Natural deduction rules of the propositional intuitionistic logic}
    \label{fig:intuitionistic-logic-natural-deduction}
\end{figure}

The proof system which consists of the sequent relation together with the deduction rules, is called \textit{natural deduction}.

The subset of IPC that only deals with $\alpha ::= x \mid \alpha \rightarrow \alpha$, with three rules of Ax, $\rightarrow$-I, and -E, is called IPC$(\rightarrow)$.