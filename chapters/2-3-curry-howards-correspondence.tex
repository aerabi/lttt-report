\section{Curry--Howard Correspondence}

The following theorem, states the Curry--Howard correspondence between implicational fragment of intuitionistic propositional logic IPC$(\rightarrow)$ and the simply typed lambda calculus $\lambda(\rightarrow)$.

\begin{theorem}
The types in $\lambda(\rightarrow)$ form an \emph{IPC}$(\rightarrow)$. Furthermore:
\begin{itemize}
    \item If $\Gamma \vdash \t : \alpha$, then $\iota_2[\Gamma] \vdash \alpha$.
    \item If $\Delta \vdash \alpha$, then there is a $\Gamma$ and a $\t$, such that $\Delta = \iota_2[\Gamma]$ and $\Gamma \vdash \t : \alpha$. \qedhere
\end{itemize}
\end{theorem}

An interesting case is when $\Gamma = \emptyset$.

\begin{corollary}
\label{corollary:theorem-if-inhabited}
A formula $\varphi$ in \emph{IPC}$(\rightarrow)$ is a true theorem, iff it's inhabited by a $\lambda(\rightarrow)$-term $\t$, that is $\vdash \t : \varphi$.
\end{corollary}

An example is the formula $(\beta \rightarrow \alpha) \rightarrow (\gamma \rightarrow \beta) \rightarrow \gamma \rightarrow \alpha$. Only the fact that the $\lambda(\rightarrow)$-term $\lambda a. \lambda b. \lambda c. a (b c)$ inhabits this type, proves it. \figref{fig:curry-howard-example} demonstrate a proof of it by natural deduction. For neatness, we named $\Gamma := a : \beta \rightarrow \alpha, b : \gamma \rightarrow \beta, c : \gamma$.

\begin{figure}
    \centering
    \begin{equation*}
\infer{\vdash \lambda a. \lambda b. \lambda c. a (b c) : (\beta \rightarrow \alpha) \rightarrow (\gamma \rightarrow \beta) \rightarrow \gamma \rightarrow \alpha}{
%
\infer{a : \beta \rightarrow \alpha \vdash \lambda b. \lambda c. a (b c) : (\gamma \rightarrow \beta) \rightarrow \gamma \rightarrow \alpha}{
%
\infer{a : \beta \rightarrow \alpha, b : \gamma \rightarrow \beta \vdash \lambda c. a (b c) : \gamma \rightarrow \alpha}{
%
\infer{a : \beta \rightarrow \alpha, b : \gamma \rightarrow \beta, c : \gamma \vdash a (b c) : \alpha}{
%
\infer{\Gamma \vdash a : \beta \rightarrow \alpha}{}
&
\infer{\Gamma \vdash b c : \beta}{
%
\infer{\Gamma \vdash b : \gamma \rightarrow \beta}{}
&
\infer{\Gamma \vdash c : \gamma}{}
}
}
}
}
}
\end{equation*}
    \caption{Proof by natural deduction that $\vdash \lambda a. \lambda b. \lambda c. a (b c) : (\beta \rightarrow \alpha) \rightarrow (\gamma \rightarrow \beta) \rightarrow \gamma \rightarrow \alpha$}
    \label{fig:curry-howard-example}
\end{figure}

A case of Corollary \ref{corollary:theorem-if-inhabited} is $\varphi = \bot$. If any term of the language inhabits the contradiction, then the underlying logic on the types is inconsistent.