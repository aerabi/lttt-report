\section{Simply Typed Lambda Calculus}

The language of simply typed lambda calculus $\lambda(\rightarrow)$ is defined by the following grammar:
\begin{align*}
    &\T ::= T \mid \T \rightarrow \T \\
    &\t ::= x \mid \lambda x. \t \mid \t_1 \t_2
\end{align*}
Here, $T$ designates constant types (take $\mathtt{Boolean} \mid \mathtt{String}$ for example) and $x$ designates variables. The $\t$'s are called \textit{terms} and the $\T$'s are called \textit{types}.

\subsection{Sequent and Natural Deduction}

A \textit{context} here is a set of ordered pairs $\ang{\t, \T}$, usually denoted by $\Gamma$. We additionally assume that there are no duplicate keys, making contexts \textit{functions} (in the set-theoretic sense). The set of keys in $\Gamma$ is denoted by $\iota_1[\Gamma]$ and the values $\iota_2[\Gamma]$. When $\iota_1[\Gamma] \cap \iota_1[\Delta] = \emptyset$, we write $\Gamma, \Delta$ for $\Gamma \cup \Delta$. We also write $\Gamma, \t: \T$ for $\Gamma, \{\ang{\t, \T}\}$.

\begin{definition}
The relation $\Gamma \vdash \t : \T$ for $\lambda(\rightarrow)$ is defined syntactically by \figref{fig:simply-typed-lambda-calculus-natural-deduction}.
\end{definition}

\begin{figure}
    \centering
    \begin{gather*}
\infer[\mbox{Var}]{\Gamma, x : \T \vdash x : \T}{}
\\ \\ 
\infer[\rightarrow\mbox{-I}]{\Gamma \vdash \lambda x. \t : \T \rightarrow \mathfrak{S}}{\Gamma, x : \T \vdash \t : \mathfrak{S}}
~~~~ ~~~~
\infer[\rightarrow\mbox{-E}]{\Gamma \vdash \t_1\t_2 : \mathfrak{S}}{
    \Gamma \vdash \t_1 : \T \rightarrow \mathfrak{S}
    &
    \Gamma \vdash \t_2 : \T
}
\end{gather*}
    \caption{Natural deduction rules of the simply typed lambda calculus}
    \label{fig:simply-typed-lambda-calculus-natural-deduction}
\end{figure}