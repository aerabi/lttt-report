\section{Structural Rules}

In natural deduction, \textit{structural rules} are inference rules that do not refer to any logical connective, but instead operates on the sequents directly.

Three common structural rules are:
\begin{itemize}
    \item \textbf{Weakening}. Which states that hypotheses of a sequent could be extended.
    \begin{gather*}
        \infer[\mbox{Weak}]{\Gamma_1, \alpha, \Gamma_2 \vdash \beta}{\Gamma_1, \Gamma_2 \vdash \beta}
        ~~~~ ~~~~
        \infer[\mbox{Weak}]{\Gamma_1, x : \A, \Gamma_2\vdash \t : \mathfrak{B}}{\Gamma_1, \Gamma_2\vdash \t : \mathfrak{B}}
    \end{gather*}
    \item \textbf{Contraction}. Which states that two similar (or unifiable) hypothesis can be replaced by a single hypothesis.
    \begin{gather*}
        \infer[\mbox{Contr}]{\Gamma_1, \alpha, \Gamma_2 \vdash \beta}{\Gamma_1, \alpha, \alpha, \Gamma_2 \vdash \beta}
        ~~~~ ~~~~
        \infer[\mbox{Contr}]{\Gamma_1, x : \A, \Gamma_2 \vdash \t : \mathfrak{B}}{\Gamma_1, x : \A, x : \A, \Gamma_2 \vdash \t : \mathfrak{B}}
    \end{gather*}
    Another version of the weakening rule is proposed by \cite{DavidWalker2004} for type systems:
    \begin{gather*}
        \infer[\mbox{Contr}]{\Gamma_1, x_1 : \A, \Gamma_2 \vdash \t[x_2 \mapsto x_1, x_3 \mapsto x_1] : \mathfrak{B}}{\Gamma_1, x_2 : \A, x_3 : \A, \Gamma_2 \vdash \t : \mathfrak{B}}
    \end{gather*}
    \item \textbf{Exchange}. Which states that two hypotheses may be swapped.
    \begin{gather*}
        \infer[\mbox{Exch}]{\Gamma_1, \alpha_2, \alpha_1, \Gamma_2 \vdash \beta}{\Gamma_1, \alpha_1, \alpha_2, \Gamma_2 \vdash \beta}
        ~~~~ ~~~~
        \infer[\mbox{Exch}]{\Gamma_1, x_2 : \A_2, x_1 : \A_1 \Gamma_2 \vdash \t : \mathfrak{B}}{\Gamma_1, x_1 : \A_1, x_2 : \A_2, \Gamma_2 \vdash \t : \mathfrak{B}}
    \end{gather*}
\end{itemize}

Any logic (or type system) that lacks any of these structural rules is called a substructural logic (or type system, respectively).

\begin{table}[!ht]
    \centering
    \begin{tabular}{l|ccc}
         & W & C & E \\
        \hline
        Unrestricted & \checkmark & \checkmark & \checkmark \\
        Affine & \checkmark &  & \checkmark \\
        Relevant &  & \checkmark & \checkmark \\
        Linear & & & \checkmark \\
        Ordered
    \end{tabular}
    \caption{Substructural logics / type systems and the structural rules they have}
    \label{tab:my_label}
\end{table}

To understand how the structural rules for the type systems affect programming languages they are used for, a list of substructural type systems is presented with the restrictions they have:
\begin{itemize}
    \item \textbf{Affine}. Every variable can be used at most once.
    \item \textbf{Relevant}. Every variable must be used at least once.
    \item \textbf{Linear}. Every variable must be used exactly once.
    \item \textbf{Ordered}. Every variable must be used exactly once and in the order in which it is introduced.
\end{itemize}