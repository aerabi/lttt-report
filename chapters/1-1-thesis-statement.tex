\section{Contributions}

\textbf{Thesis Statement.}

\textit{Linear-time temporal type theory is a natural model for event-driven programming. The theory is sound and has practical applications in a variety of programming languages and libraries.}

To support this statement, this dissertation makes the following contributions:

\begin{itemize}
    \item \chapref{chap:background} contains a tutorial and novel formulation of theoretical background to set a basis for event-driven programming. This includes:
    \begin{itemize}
        \item A crash course on intuitionistic logic, simply typed lambda calculus, linear lambda calculus, linear logic, and modal logic,
        \item Stating Curry--Howard correspondence,
        \item A review and reformulation of (linear-time) temporal logic.
    \end{itemize}
    \item \chapref{chapter:the-type-system} formulates the linear-time temporal type theory, LTTT, based on work of \cite{Paykin2016TheEO}, and discusses its connection to the real-world event-driven programming paradigm.
    \item \chapref{chap:coq-implementation} presents the implementation of LTTT in Coq, to verify the theory's soundness. This includes:
    \begin{itemize}
        \item A Coq implementation of generic environments to be used in the typing rules as a linear and non-linear context,
        \item A Coq implementation of typing rules and operational semantics of LTTT,
        \item A set of proved lemmas and propositions that show interesting results or are necessary for the other theorems,
        \item A mechanical proof of preservation.
    \end{itemize}
    \item \chapref{chap:typescript-implementation} presents an implementation of LTTT in TypeScript, to demonstrate LTTT in practice, and to set an infrastructure for further discussion. The TypeScript implementation is published as an NPM library and can be used in production code.
\end{itemize}