\chapter{Introduction}\label{chap:introduction}

\section{Problem Definition}

Event-driven programming is becoming the standard paradigm for programming in almost every domain: backend, frontend, GUI, etc. All of the framework and library vendors are now offering \textit{reactive}, \textit{i.e.} event-driven interfaces for their products. Spring framework came up with WebFlux, Angular is all wrapped in RxJS's observables, and Node.js is async by nature. Reactive drivers are being implemented for databases, to support an end-to-end reactive spectrum.

In such a world, laying a basis for event-driven programming is necessary more than ever. \cite{Paykin2016TheEO} introduced a type system based on linear/non-linear logic of \cite{DBLP:conf/csl/Benton94} and linear-time temporal logic of \cite{DBLP:conf/focs/Pnueli77} and inspired by the linear logic of \cite{DBLP:journals/tcs/Girard87}. We will refer to this type system and the theory around it as linear-time temporal type theory, or LTTT for short.

We will explain and examine LTTT from different perspectives. A Coq implementations tries to do a mechanical examination of the theory, and a TypeScript implementation and discussion tries to validate the theory in action.

\subsection{Generic Store}

Typing contexts lie at the core of type inference relations, storing all the hypotheses of a sequent. Having the right data model for the context is specially important as it will affect your typing rules as well as the proofs about them.

Among his libraries for the locally nameless infrastructure, Arthur \cite{DBLP:journals/jar/Chargueraud12} built a generic model for environments, and later Emmanuel \cite{DBLP:journals/corr/abs-1112-1316} rewrote his generic environments model in Coq 8.3, publishing it as a standalone Coq library, namely \texttt{generic-environments}\footnote{\url{https://github.com/coq-community/generic-environments}}.

The initial problem with \texttt{generic-environments} was incompatibility with newer versions of Coq. After discussions with the library's author and Th\'eo Zimmermann, the library was moved to \texttt{coq-community} and then published for the newer versions of Coq by the author of current thesis.

Still, the implementation of \texttt{generic-environments} made use of Coq features that had breaking changes in the newer versions, hence a new simpler model was implemented.

\subsection{TypeScript Embedding}

Linear and affine types are not supported by many mainstream programming languages. \cite{JenniferPaykin2018} worked on embedding linear types in non-linear host languages. She particularly implemented an embedding for Coq and Haskell.

To study event-driven programming in action, we chose JavaScript as it is the most popular programming language. Furthermore, the server-side JS is async by nature, and hence most of the logic written in Node.JS are wrapped into callbacks or async types such as \texttt{Promise} and \texttt{Observable}.

TypeScript is currently the typed version JavaScript, and hence was used for our study. But, there are fundamental differences between TypeScript and Haskell or Coq. Types in TypeScript are parts of the metalanguage and are not shipped to the executable code. Hence, one has very limited access to types in the runtime and particularly cannot store them in a key-value store.

So, to embed linearity into TypeScript/JavaScript, one should employ logic that has similar behaviour.

\section{Contributions}

\textbf{Thesis Statement.}

\textit{Linear-time temporal type theory is a natural model for event-driven programming. The theory is sound and has practical applications in a variety of programming languages and libraries.}

To support this thesis, this dissertation makes the following contributions:

\begin{itemize}
    \item \chapref{chap:background} contains a tutorial and novel formulation of theoretical background for setting a basis for event-driven programming. This includes:
    \begin{itemize}
        \item A crash course on intuitionistic logic, simply typed lambda calculus, linear lambda calculus, linear logic, and modal logic,
        \item Stating Curry--Howard correspondence,
        \item A review and reformulation of (linear-time) temporal logic.
    \end{itemize}
    \item \chapref{chapter:the-type-system} formulates the linear-time temporal type theory, LTTT, based on work of \cite{Paykin2016TheEO}, and discusses its connection to the real-world event-driven programming paradigm.
    \item \chapref{chap:coq-implementation} presents the implementation of LTTT in Coq, to verify that the theory's soundness. This includes:
    \begin{itemize}
        \item A Coq implementation of generic environments to be used in the typing rules as a linear and non-linear context,
        \item A Coq implementation of typing rules and operational semantics of LTTT,
        \item A set of proved lemmas and propositions that show interesting results or are necessary for the other theorems,
        \item A mechanical proof of preservation.
    \end{itemize}
    \item \chapref{chap:typescript-implementation} presents an implementation of LTTT in TypeScript, to demonstrate LTTT in practice, and to set an infrastructure for further discussion. The TypeScript implementation is published as an NPM library and can be used in production code.
\end{itemize}
\section{Conventions}

Prior familiarity with logic and type systems is helpful but not required. \chapref{chap:background} tries to define everything required to understand the LTTT's formulation later given in \chapref{chapter:the-type-system}. For more background in type systems, we refer the reader to the book of \cite{10.5555/1076265} and especially its first chapter by \cite{DavidWalker2004}. For more background in the theoretical part, work of \cite{DBLP:journals/tcs/Girard87} on linear logic is suggested.

Throughout the thesis, small Greek letters are used as metavariables in the logics: $\alpha, \beta, \varphi, \psi$. In the type systems, Fraktur letters are used instead: small letters for terms and expressions in the language being typed, e.g. $\t, \e$, and capital letters for the types: $\mathfrak{A}, \mathfrak{B}, \T$. One reason for this separation is because some small Greek letters are used in the toy languages: $\lambda$ for the lambda expression, $\iota$ for injection, and $\pi$ for projection. Furthermore, most capital Greek letters are indistinguishable from their Latin descendants, e.g. $A, B, T$. The use of Fraktur letters for logical expressions is inspired by \cite{hilbert1928}.
In \chapref{chap:coq-implementation}, the Fraktur letters represent types in the metalanguage (in this case Coq):

\begin{minted}[escapeinside=<>,mathescape=true]{coq}
Inductive type : <$\mathfrak{G}$> -> <$\mathfrak{t}$> -> <$\T$> -> Prop := ...
\end{minted}

\begin{table}[t]
    \centering
    \begin{tabular}{l|l|p{0.6\linewidth}}
        Example & Name & Meaning  \\
        \hline
        $\alpha\beta\gamma$ & Greek small & metavariables in logics \\
        $\Gamma\Delta$ & Greek capital & (typing) contexts \\
        $\mathfrak{a}\mathfrak{b}\mathfrak{c}$ & Fraktur small & metavariables ranging over programming languages in type systems \\
        $\A\mathfrak{B}\mathfrak{C}$ & Fraktur capital & metavariables ranging over types in type systems \\
        \texttt{abc} & monospace & code
    \end{tabular}
    \caption{Different typefaces and their meanings}
    \label{tab:fonts}
\end{table}

Here \texttt{type} is the typing relation represented as a sequent in \chapref{chapter:the-type-system}: $\Gamma \vdash \t : \T$.

\chapref{chap:coq-implementation} discusses a Coq implementation of the theory, so familiarity with the syntax of Coq is helpful. Similarly, familiarity with syntax of TypeScript is helpful for \chapref{chap:typescript-implementation}.