\chapter{Introduction}\label{chap:introduction}

In this work, we implemented the linear-time temporal type theory, LTTT, in Coq. LTTT is first introdcued by \cite{Paykin2016TheEO} to lay a basis to study event-driven programming. Particularly, we implemented typing rules and an operational semantics, and proved some results about them (among them some stronger linear and non-linear preservation results) mechanically in Coq. We implemented a generic store in Coq that is used as the linear and non-linear context for the typing relations. Some 45 lemmas and propositions are proved about the store to make it more practical. And finally, we implemented an embedded version of LTTT in TypeScript to show the theory in action. The TypeScript implementation tries to embed linearity into the non-linear type system of TS and create a controlled version of RxJS's observable.

\section{Motivation}

Event-driven programming is becoming the standard paradigm for programming in almost every domain: backend, frontend, GUI, etc. All of the framework and library vendors are now offering \textit{reactive}, \textit{i.e.} event-driven interfaces for their products. Spring framework came up with WebFlux, Angular is all wrapped in RxJS's observables, and Node.js is async by nature. Reactive drivers are being implemented for databases, to support an end-to-end reactive spectrum.

Unfortunately, event-driven programming can be difficult, as it is a combination of two seemingly contradictory elements: callbacks and states. When callbacks change the state of a program, we say they have \textit{side effects}, a very unpopular term among developers (e.g. see \cite{aerabi_2020}).

To overcome these problems, many different and somewhat ad hoc approaches were taken. Scala, as an example, has at least 3 different built-in abstractions for event types: \texttt{Promise}, \texttt{Future}, and \texttt{Signal} (\cite{deprecating2010}), putting RxScala aside. And of course we have access to Java libraries in Scala, adding Reactor and similar abstractions to the bunch.

In such a time, laying a basis for event-driven programming seems necessary more than ever.

This work aims to study event-driven programming by identifying structures common to it with a natural perspective from temporal logic. Under this correspondence, an event (specially in the Rx or \texttt{Promise} sense) is computation that can eventually return a value, and can be identified with the temporal logic's \textit{eventually}/\textit{future} modality $\diamond\A$.

A correspondence between linear-time temporal logic (LTL) and event-driven programming is presented in terms of Curry--Howard correspondence, and then is tried to be verified with an implementation in the proof-assistant Coq language (\cite{mohammad_ali_a_rabi_2021_4749276}). A novel reformulation of the theory behind LTL and Curry--Howard correspondence is also presented. And finally a TypeScript implementation tries to relate the theory to real-life production code written in JavaScript/TypeScript.

% \subsection{TypeScript Embedding}

% Linear and affine types are not supported by many mainstream programming languages. \cite{JenniferPaykin2018} worked on embedding linear types in non-linear host languages. She particularly implemented an embedding for Coq and Haskell.

% To study event-driven programming in action, we chose JavaScript for multiple reasons. To name one, the server-side JS is async by nature, and hence most of the logic written in Node.JS are wrapped into callbacks or async types such as \texttt{Promise} and \texttt{Observable}.

% TypeScript is currently the typed version of JavaScript and hence was used for our study. But, there are fundamental differences between TypeScript and Haskell or Coq. Types in TypeScript are parts of the metalanguage and are not shipped to the executable code. Hence, one has very limited access to types in the runtime and particularly cannot store them in a key-value store.

% So, to embed linearity into TypeScript/JavaScript, one should employ logic that has similar behavior.

\section{Thesis Layout}

\textbf{Thesis Statement.}

\textit{Linear-time temporal type theory is a natural model for event-driven programming. The theory is sound and has practical applications in a variety of programming languages and libraries.}

To support this statement, this dissertation makes the following contributions:

\begin{itemize}
    \item \chapref{chap:background} contains a tutorial and novel formulation of the theoretical background to set a basis for event-driven programming. This includes:
    \begin{itemize}
        \item A crash course on intuitionistic logic, simply typed lambda calculus, linear lambda calculus, linear logic, and modal logic,
        \item Stating Curry--Howard correspondence,
        \item A review and reformulation of (linear-time) temporal logic.
    \end{itemize}
    \item \chapref{chapter:the-type-system} formulates the linear-time temporal type theory, LTTT, based on work of \cite{Paykin2016TheEO}, and discusses its connection to the real-world event-driven programming paradigm.
    \item \chapref{chap:coq-implementation} presents our Coq implementation of LTTT partially. This includes:
    \begin{itemize}
        \item A representation of Coq implementation of the generic store which used in the typing rules as a linear and non-linear context,
        \item A representation of Coq implementation of typing rules and operational semantics of LTTT,
        \item A set of proved lemmas and propositions that show interesting results or are necessary for the other theorems,
        \item A representation of the mechanical proof of preservation.
    \end{itemize}
    \item \chapref{chap:typescript-implementation} presents an implementation of LTTT in TypeScript, to demonstrate LTTT in practice, and to set an infrastructure for further discussion. The TypeScript implementation is published as an NPM library and can be used in production code.
\end{itemize}
\section{Audience}

This dissertation can be read by undergraduate students of computer science and mathematics.
Prior familiarity with logic and type systems is helpful but not required. \chapref{chap:background} tries to define everything required to understand the LTTT's formulation later given in \chapref{chapter:the-type-system}. For more background in the logic used in the current work, we refer the reader to the book of \cite{ben-ari2012book} and its chapter on linear-time temporal logic, \cite{Ben-Ari2012}. For more background in type systems, we refer the reader to the book of \cite{10.5555/1076265} and especially its first chapter by \cite{DavidWalker2004} on linear type systems.
For more background in the theoretical part, work of \cite{DBLP:journals/tcs/Girard87} on linear logic is suggested.
The Backus--Naur form (BNF) is used extensively to define formal languages (see \cite{DBLP:conf/ifip/Backus59}).

To describe the future types, we draw examples from ReactiveX\footnote{\url{http://reactivex.io/}} and especially RxJS\footnote{\url{https://github.com/ReactiveX/rxjs}}, as well as Promise/A specification (\cite{zyp_2010}) and its implementation in JavaScript (\cite{bershanskiy_mills_willee_ribaric_2020}).

\section{Conventions}

Throughout this dissertation, small Greek letters are used as metavariables in the logics: $\alpha, \beta, \varphi, \psi$. In the type systems, Fraktur letters are used instead: small letters for terms and expressions in the language being typed, e.g. $\t, \e$, and capital letters for the types: $\mathfrak{A}, \mathfrak{B}, \T$. One reason for this separation is because some small Greek letters have meanings in standard formulation of programming languages and therefore are used in the toy languages: $\lambda$ for the lambda expression, $\iota$ for injection, and $\pi$ for projection. Furthermore, most capital Greek letters are indistinguishable from their Latin descendants, e.g. $A, B, T$. The use of Fraktur letters for logical expressions is inspired by \cite{hilbert1928}.
In \chapref{chap:coq-implementation}, the Fraktur letters represent types in the metalanguage (in this case Coq):

\begin{minted}[escapeinside=<>,mathescape=true]{coq}
Inductive type : <$\mathfrak{G}$> -> <$\mathfrak{t}$> -> <$\T$> -> Prop := ...
\end{minted}

\begin{table}[t]
    \centering
    \begin{tabular}{l|l|p{0.6\linewidth}}
        Example & Name & Meaning  \\
        \hline
        $\alpha\beta\gamma$ & Greek small & metavariables in logics \\
        $\Gamma\Delta$ & Greek capital & (typing) contexts \\
        $\mathfrak{a}\mathfrak{b}\mathfrak{c}$ & Fraktur small & metavariables ranging over programming languages in type systems \\
        $\A\mathfrak{B}\mathfrak{C}$ & Fraktur capital & metavariables ranging over types in type systems \\
        \texttt{abc} & monospace & code
    \end{tabular}
    \caption{Different typefaces and their meanings}
    \label{tab:fonts}
\end{table}

Here \texttt{type} is the typing relation represented as a sequent in \chapref{chapter:the-type-system}: $\Gamma \vdash \t : \T$.