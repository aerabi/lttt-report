\section{Audience}

This dissertation can be read by undergraduate students of computer science and mathematics.
Prior familiarity with logic and type systems is helpful but not required. \chapref{chap:background} tries to define everything required to understand the LTTT's formulation later given in \chapref{chapter:the-type-system}. For more background in the logic used in the current work, we refer the reader to the book of \cite{ben-ari2012book} and its chapter on linear-time temporal logic, \cite{Ben-Ari2012}. For more background in type systems, we refer the reader to the book of \cite{10.5555/1076265} and especially its first chapter by \cite{DavidWalker2004} on linear type systems.
For more background in the theoretical part, work of \cite{DBLP:journals/tcs/Girard87} on linear logic is suggested.
The Backus--Naur form (BNF) is used extensively to define formal languages (see \cite{DBLP:conf/ifip/Backus59}).

To describe the future types, we draw examples from ReactiveX\footnote{\url{http://reactivex.io/}} and especially RxJS\footnote{\url{https://github.com/ReactiveX/rxjs}}, as well as Promise/A specification (\cite{zyp_2010}) and its implementation in JavaScript (\cite{bershanskiy_mills_willee_ribaric_2020}).

\section{Conventions}

Throughout this dissertation, small Greek letters are used as metavariables in the logics: $\alpha, \beta, \varphi, \psi$. In the type systems, Fraktur letters are used instead: small letters for terms and expressions in the language being typed, e.g. $\t, \e$, and capital letters for the types: $\mathfrak{A}, \mathfrak{B}, \T$. One reason for this separation is because some small Greek letters have meanings in standard formulation of programming languages and therefore are used in the toy languages: $\lambda$ for the lambda expression, $\iota$ for injection, and $\pi$ for projection. Furthermore, most capital Greek letters are indistinguishable from their Latin descendants, e.g. $A, B, T$. The use of Fraktur letters for logical expressions is inspired by \cite{hilbert1928}.
In \chapref{chap:coq-implementation}, the Fraktur letters represent types in the metalanguage (in this case Coq):

\begin{minted}[escapeinside=<>,mathescape=true]{coq}
Inductive type : <$\mathfrak{G}$> -> <$\mathfrak{t}$> -> <$\T$> -> Prop := ...
\end{minted}

\begin{table}[t]
    \centering
    \begin{tabular}{l|l|p{0.6\linewidth}}
        Example & Name & Meaning  \\
        \hline
        $\alpha\beta\gamma$ & Greek small & metavariables in logics \\
        $\Gamma\Delta$ & Greek capital & (typing) contexts \\
        $\mathfrak{a}\mathfrak{b}\mathfrak{c}$ & Fraktur small & metavariables ranging over programming languages in type systems \\
        $\A\mathfrak{B}\mathfrak{C}$ & Fraktur capital & metavariables ranging over types in type systems \\
        \texttt{abc} & monospace & code
    \end{tabular}
    \caption{Different typefaces and their meanings}
    \label{tab:fonts}
\end{table}

Here \texttt{type} is the typing relation represented as a sequent in \chapref{chapter:the-type-system}: $\Gamma \vdash \t : \T$.