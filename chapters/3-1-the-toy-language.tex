\section{Event-Based Toy Language and Types}

Throughout this chapter, a toy language together with a type system is used to discuss using temporal logic in event-driven programming. The type system is presented in the work of \cite{Paykin2016TheEO}. The types are based on the linear/non-linear logic of \cite{DBLP:conf/csl/Benton94} with an addition of $\diamond$ modality.

The language and its type are defined in \figref{fig:syntax-of-types-terms-expressions}. Non-linear types are designated with $\T$ and linear types are designated with $\A$. Terms, denoted by $\t$, form the non-linear fragment of the language. On the other hand, expressions, denoted by $\e$, form the linear fragment.

\begin{figure}
    \centering
    \begin{align*}
    \T ::=&~ \mathtt{Unit} \mid \mathtt{Void} \mid \T \times \T \mid \T + \T \mid \T \rightarrow \T \mid \ceil{\A} \\
    \A ::=&~ 1 \mid 0 \mid \A \otimes \A \mid \A \otimes \A \mid \A \multimap \A \mid \diamond\A \mid \floor{\T} \\
    \\
    \t ::=&~ x \mid () \mid \mathtt{case}~\t~\mathtt{of}~() \mid (\t_1, \t_2) \mid \pi_i \t \mid \mathtt{in}_i \t \mid \mathtt{case}~\t~\mathtt{of}~(\mathtt{in}_1 x_1 \rightarrow \t_1 \mid \mathtt{in}_2 x_2 \rightarrow \t_2) \\
    & \mid \lambda x. \t \mid \t_1 \t_2 \mid \mathtt{suspend}~\e \\
    \e ::=&~ x \mid () \mid \mathtt{case}~\e~\mathtt{of}~() \mid (\e_1, \e_2) \mid \mathtt{in}_i \e \mid \mathtt{case}~\t~\mathtt{of}~(\mathtt{in}_1 x_1 \rightarrow \e_1 \mid \mathtt{in}_2 x_2 \rightarrow \e_2) \\
    & \mid \lambda x. \e \mid \e_1 \e_2 \mid \mathtt{force}~\t \mid \mathtt{return}~\e \mid \mathtt{bind}~x = \e_1~\mathtt{in}~\e_2 \mid \floor{\t} \\
    &\mid \mathtt{let}~() = \e_1~\mathtt{in}~\e_2 \mid \mathtt{let}~(x_1, x_2) = \e_1~\mathtt{in}~\e_2 \mid \mathtt{let}~\floor{x} = \e_1~\mathtt{in}~\e_2
\end{align*}


    \caption{Syntax of linear and non-linear types, terms, and expressions}
    \label{fig:syntax-of-types-terms-expressions}
\end{figure}

\begin{figure}
    \centering
    \begin{gather*}
\infer[\mbox{Var}]{\Gamma, x: \T \vdash x : \T}{}
~~~~ ~~~~
\infer[\mathtt{Unit}\mbox{-I}]{\Gamma \vdash () : \mathtt{Unit}}{}
~~~~ ~~~~
\infer[\mathtt{Void}\mbox{-E}]{\Gamma \vdash \mathtt{case}~\t~\mathtt{of}~() : \mathfrak{S}}{\Gamma \vdash \t : \mathtt{Void}}
\\ \\ 
\infer[\times\mbox{-I}]{\Gamma \vdash (\t_1, \t_2) : \T_1 \times \T_2}{\Gamma \vdash \t_1 : \T_1 & \Gamma \vdash \t_2 : \T_2}
~~~~ ~~~~
\infer[\times\mbox{-E}]{\Gamma \vdash \pi_i \t : \T_i}{\Gamma \vdash \t : \T_1 \times \T_2}
\\ \\ 
\infer[+\mbox{-I}]{\Gamma \vdash \mathtt{in}_i \t : \T_1 + \T_2}{\Gamma \vdash \t : \T_i}
~~~~ ~~~~
\infer[+\mbox{-E}]{\Gamma \vdash \mathtt{case}~\t~\mathtt{of}~(\mathtt{in}_1 x_1 \rightarrow \t_1 \mid \mathtt{in}_2 x_2 \rightarrow \t_2) : \mathfrak{S}}{
    \Gamma \vdash \t : \T_1 + \T_2
    &
    \Gamma, x_1 : \T_1 \vdash \t_1 : \mathfrak{S}
    &
    \Gamma, x_2 : \T_2 \vdash \t_2 : \mathfrak{S}
}
\\ \\
\infer[\rightarrow\mbox{-I}]{\Gamma \vdash \lambda x. \t : \T \rightarrow \mathfrak{S}}{\Gamma, x : \T \vdash \t : \mathfrak{S}}
~~~~ ~~~~
\infer[\rightarrow\mbox{-E}]{\Gamma \vdash \t_1\t_2 : \mathfrak{S}}{
    \Gamma \vdash \t_1 : \T \rightarrow \mathfrak{S}
    &
    \Gamma \vdash \t_2 : \T
}
\end{gather*}

    \caption{Typing rules of the linear types}
    \label{fig:typing-rules-for-nonlinear-types}
\end{figure}

\begin{figure}
    \centering
    \begin{gather*}
\infer[\mbox{Var}]{\Gamma, x: \A \vdash x : \A}{}
~~~~ ~~~~
\infer[0\mbox{-E}]{\Gamma; \Delta \vdash \mathtt{case}~\e~\mathtt{of}~() : \mathfrak{B}}{\Gamma; \Delta \vdash \e : 0}
\\ \\ 
\infer[1\mbox{-I}]{\Gamma; \cdot \vdash () : 1}{}
~~~~ ~~~~
\infer[1\mbox{-E}]{\Gamma; \Delta_1, \Delta_2 \vdash \mathtt{let}~() = \e_1~\mathtt{in}~\e_2 : \mathfrak{B}}{
    \Gamma; \Delta_1 \vdash \e_1 : 1
    &
    \Gamma; \Delta_2 \vdash \e_2 : \mathfrak{B}
}
\\ \\ 
\infer[\otimes\mbox{-I}]{\Gamma; \Delta_1, \Delta_2 \vdash (\e_1, \e_2) : \A_1 \otimes \A_2}{
    \Gamma; \Delta_1 \vdash \e_1 : \A_1
    &
    \Gamma; \Delta_2 \vdash \e_2 : \A_2
}
\\ \\ 
\infer[\otimes\mbox{-E}]{\Gamma; \Delta_1, \Delta_2 \vdash \mathtt{let}~(x_1, x_2) = \e_1~\mathtt{in}~\e_2 : \mathfrak{B}}{
    \Gamma; \Delta_1 \vdash \e_1 : \A_1 \otimes \A_2
    &
    \Gamma; \Delta_2, x_1 : \A_1, x_2 : \A_2 \vdash \e_2 : \mathfrak{B}
}
\\ \\ 
\infer[\oplus\mbox{-I}]{\Gamma; \Delta \vdash \mathtt{in}_i \e : \A_1 \oplus \A_2}{\Gamma; \Delta \vdash \e : \A_i}
\\ \\
\infer[\oplus\mbox{-E}]{\Gamma; \Delta_1, \Delta_2 \vdash \mathtt{case}~\t~\mathtt{of}~(\mathtt{in}_1 x_1 \rightarrow \e_1 \mid \mathtt{in}_2 x_2 \rightarrow \e_2) : \mathfrak{B}}{
    \Gamma; \Delta_1 \vdash \e : \A_1 \oplus \A_2
    &
    \Gamma; \Delta_2, x_1 : \A_1 \vdash \e_1 : \mathfrak{B}
    &
    \Gamma; \Delta_2, x_2 : \A_2 \vdash \e_2 : \mathfrak{B}
}
\\ \\ 
\infer[\multimap\mbox{-I}]{\Gamma; \Delta \vdash \lambda x. \e : \A \multimap \mathfrak{B}}{\Gamma; \Delta, x : \A \vdash \e : \mathfrak{B}}
~~~~ ~~~~
\infer[\multimap\mbox{-E}]{\Gamma; \Delta_1, \Delta_2 \vdash \e_1\e_2 : \mathfrak{B}}{
    \Gamma; \Delta_1 \vdash \e_1 : \A \multimap \mathfrak{B}
    &
    \Gamma; \Delta_2 \vdash \e_2 : \A
}
\\ \\ 
\infer[\diamond\mbox{-I}]{\Gamma; \Delta \vdash \mathtt{return}~\e : \diamond\A}{\Gamma; \Delta \vdash \e : \A}
~~~~ ~~~~
\infer[\diamond\mbox{-E}]{\Gamma; \Delta_1, \Delta_2 \vdash \mathtt{bind}~x = \e_1~\mathtt{in}~\e_2 : \diamond\mathfrak{B}}{
    \Gamma; \Delta_1 \vdash \e_1 : \diamond\A
    &
    \Gamma; \Delta_2, x : \A \vdash \e_2 : \diamond\mathfrak{B}
}
\end{gather*}
    \caption{Typing rules of the non-linear types}
    \label{fig:typing-rules-for-linear-types}
\end{figure}

\begin{figure}
    \centering
    \begin{gather*}
\infer[\ceil{\cdot}\mbox{-I}]{\Gamma \vdash \mathtt{suspend}~\e : \ceil{\A}}{\Gamma; \cdot \vdash \e : \A}
~~~~ ~~~~
\infer[\ceil{\cdot}\mbox{-E}]{\Gamma; \cdot \vdash \mathtt{force}~\t : \A}{\Gamma \vdash \t : \ceil{\A}}
\\ \\ 
\infer[\floor{\cdot}\mbox{-I}]{\Gamma; \cdot \vdash \floor{\t} : \floor{\T}}{\Gamma \vdash \t : \T}
~~~~ ~~~~
\infer[\floor{\cdot}\mbox{-E}]{\Gamma; \Delta_1, \Delta_2 \vdash \mathtt{let}~\floor{x} = \e_1~\mathtt{in}~\e_2 : \mathfrak{B}}{
    \Gamma; \Delta_1 \vdash \e_1 : \floor{\T}
    &
    \Gamma, x : \T; \Delta_2 \vdash \e_2 : \mathfrak{B}
}
\end{gather*}
    \caption{Typing rules for moving between linear and non-linear types}
    \label{fig:typing-rules-for-linear-and-nonlinear-types}
\end{figure}

\begin{itemize}
    \item $\mathtt{Void}$ (resp. $0$) represents a type that is inhabited by no term (resp. expression).
    \item $\mathtt{Unit}$ (resp. $1$) represents a type that is inhabited by exactly one term (resp. expression), namely $()$ (resp. $()$). We won't differentiate the term $()$ and the expression $()$ in notation, as it should be apparent from the context which one we are talking about.
    \item The multiplicative connectives $\times$ and $\otimes$ represent type of pairs, in the non-linear and linear domains, respectively.
    \item The additive connectives $+$ and $\oplus$ are used for branching.
    \item The connectives $\rightarrow$ and $\multimap$ represent the type of functions.
    \item The modality $\diamond$ represent the type of a variable that will eventually become available.
\end{itemize}

\begin{definition}
Two sequent relations are introduced as follows:
\begin{itemize}
    \item The linear sequent $\Gamma; \Delta \vdash \e : \A$ which is defined through \figref{fig:typing-rules-for-linear-types},
    \item The non-linear sequent $\Gamma \vdash \t : \T$ which is defined through \figref{fig:typing-rules-for-nonlinear-types},
    \item The relation between the two sequents are given by \figref{fig:typing-rules-for-linear-and-nonlinear-types}.
\end{itemize}
Here, $\Gamma$ denotes the unrestricted typing context, and $\Delta$ the linear context. The non-linear sequent has access only to the unrestricted context, although the linear sequent has access to both.
\end{definition}