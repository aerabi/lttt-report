\section{Linear-Time Temporal Logic}

Linear-time temporal logic (LTL) is a modal temporal logic reasoning about the linear time, in contrast to the branching-time temporal logics such as computation tree logic (CTL). LTL was introduced by \cite{DBLP:conf/focs/Pnueli77}.

LTL adds two more operators to the temporal logic:
\begin{itemize}
    \item Weak until. $\varphi~W~\psi$, or $\varphi~\varheartsuit~\psi$, ``$\varphi$ (holds at least) until $\psi$ (and $\varphi$ holds forever if $\psi$ never becomes true)''.
    \item Strong release. $\varphi~M~\psi$, or $\varphi~\vardiamondsuit~\psi$, ``$\psi$ is true until and including the point where $\varphi$ becomes true (and $\varphi$ has to hold eventually)''.
\end{itemize}

The following equivalences hold in LTL:
\begin{itemize}
    \item Next is self-dual: $\neg\bigcirc\varphi \equiv \bigcirc\neg\varphi$.
    \item Future/eventually and always are dual:
    \begin{eqnarray*}
    \neg\diamond\varphi &\equiv& \square\neg\varphi, \\
    \neg\square\varphi &\equiv& \diamond\neg\varphi.
    \end{eqnarray*}
    \item Until and release are dual: 
    \begin{eqnarray*}
    \neg(\varphi~\spadesuit~\psi) &\equiv& \neg\varphi~\clubsuit~\neg\psi, \\
    \neg(\varphi~\clubsuit~\psi) &\equiv& \neg\varphi~\spadesuit~\neg\psi.
    \end{eqnarray*}
    \item Weak until and strong release are dual: 
    \begin{eqnarray*}
    \neg(\varphi~\varheartsuit~\psi) &\equiv& \neg\varphi~\vardiamondsuit~\neg\psi, \\
    \neg(\varphi~\vardiamondsuit~\psi) &\equiv& \neg\varphi~\varheartsuit~\neg\psi.
    \end{eqnarray*}
\end{itemize}

Some special temporal axioms in LTL are the following:
\begin{itemize}
    \item $\square\varphi \rightarrow \square\square\varphi$ (\textbf{4}).
    \item $\square\square\varphi \rightarrow \square\varphi$ (follows from \textbf{T}).
    \item $\diamond\varphi \rightarrow \diamond\diamond\varphi$ (follows from \textbf{5} and \textbf{D}). 
    \item $\diamond\diamond\varphi \rightarrow \diamond\varphi$.
\end{itemize}

% \begin{theorem}
% For formulae $\varphi$ and $\psi$ we have: $\diamond\varphi \wedge (\varphi \rightarrow \diamond\psi) \rightarrow \diamond\psi$.
% \end{theorem}
% \begin{proof}
% \end{proof}