\section{Embedding Linear Types in TypeScript}

The \texttt{Observable} type of RxJS is already a very good model to study linear events, but it is too complicated. We need a more limited and controlled version of an \texttt{Observable}. In particular, we need our new monadic type to satisfy the following criteria:
\begin{itemize}
    \item Wrap exactly one event (just like \texttt{Promise} or \texttt{Single} of RxJava).
    \item Expose one binding interface (similar to \texttt{Promise}'s \texttt{next}), rather than dozens like in \texttt{Observable}.
    \item Be linear (unlike \texttt{Promise} that can be used indefinitely).
\end{itemize}

% Embedding linear type systems in non-linear host languages was studied by \cite{JenniferPaykin2018}. In this chapter we describe the implementation of the type system presented in Chapter \ref{chapter:the-type-system} in TypeScript as the host language.

The key ingredients to create such a type system are the following:
\begin{itemize}
    \item A class \texttt{Lin<}$\T$\texttt{>} that implements $\floor{\T}$, to create an embedded linear type system into TypeScript, that lacks it.
    \item A class that implements $\diamond\A$, based on a \texttt{Lin}.
\end{itemize}

The explanations that follow are based on the TypeScript implementation of \texttt{Lin}, \texttt{LPromise}, and \texttt{Single}\footnote{\url{https://github.com/aerabi/lttt-typescript}}.

\section{Linear Type of \texttt{Lin}}
Type \texttt{Lin} is implemented as a TypeScript class.
One can create a linear variable from a non-linear value by calling on the constructor of \texttt{Lin}:

\begin{minted}{typescript}
const lin: Lin<number> = new Lin(12);
\end{minted}

And to read the value later, one needs to call on the class method \texttt{read}:

\begin{minted}{typescript}
console.log(lin.read());
\end{minted}

The \texttt{read} method cannot be called more than once; it will through an error on the second attempt. And code linters such as TSLint and ESLint can guard against the variable not being used.

The types \texttt{LUnit} and \texttt{LZero} are special cases of the type \texttt{Lin}:

\begin{minted}{typescript}
export type LUnit = Lin<undefined>;
export type LZero = Lin<void>;
\end{minted}

\section{Eager Type of \texttt{LPromise}}

As presented in \chapref{chapter:the-type-system}, the modal $\diamond$ exists in the linear fragment of the logic. In our setting, we need to use \texttt{Lin} to convert an unrestricted type into a linear one, and then apply the modality on it.

This section explains the type \texttt{LPromise<}$\T$\texttt{>}, which is an implementation of $\diamond\floor{\T}$ (so the generic unrestricted type is converted into a linear type under the hood):

\begin{minted}{typescript}
let lpromise: LPromise<string>;
\end{minted}

The \texttt{LPromise} type wraps a linear value that is not necessarily available immediately. One can create a single from a value or a \texttt{Promise}:

\begin{minted}{typescript}
const lp1 = LPromise.fromLinearValue(lin);
const lp2 = LPromise.fromValue(12);
const lp3 = LPromise.fromPromise(Promise.resolve(12));
\end{minted}

The values are wrapped into a linear variable under the hood, so all of the 3 introduced variables (\texttt{lp1} and \texttt{lp2} and \texttt{lp3}) are equal.

There is a \texttt{bind} method that will read the value of the eventual linear variable and pass it to the callback function:

\begin{minted}{typescript}
const lpromise = LPromise.fromValue(12);
lpromise.bind(x => expect(x).toEqual(12));
\end{minted}

And, in contrast to \texttt{Observable}s and \texttt{Promise}s, a \texttt{Single} cannot be bound twice:

\begin{minted}{typescript}
lpromise.bind(x => ...);  // first time, OK
lpromise.bind(x => ...);  // second time, error
\end{minted}

\section{Lazy Type of \texttt{Single}}

A \texttt{Single} is very similar to an \texttt{LPromise}, except for its laziness: The callbacks are only computed when the \texttt{exec} method is called.

\begin{minted}{typescript}
const single = Single.fromValue(12);
single.bind(console.log);  // nothing happens...
single.exec(() => {});     // now, it prints to console
\end{minted}


\begin{listing}[t]
\begin{minted}[
frame=lines,
framesep=2mm,
baselinestretch=1.2,
fontsize=\footnotesize,
linenos,
mathescape=true
]{typescript}
import { Single } from 'lttt';

export class UserRepository {
  constructor() {
    // initialize the DB model
  }
  
  public getUserById(id: string): Single<User> {
    return Single.fromPromise(model.getById(id))
      .bind(UserRepository.toBusinessModel);
  }
  
  private static toBusinessModel(doc: UserDocument): User {
    return { name: doc.name, id: doc._id };
  }
}
\end{minted}
\caption{The \texttt{Single} type in a action}
\label{listing:user-repository}
\end{listing}

An example of the \texttt{Single} type being used in action is demonstrated in Listing \ref{listing:user-repository}. Database model returns a value of type \texttt{Promise<UserDocument>} which is cast into \texttt{Single<UserDocument>}. The method \texttt{fromPromise} transforms \texttt{UserDocument} into \texttt{Lin<UserDocument>} under the hood and wraps it in the \texttt{Single}.

The \texttt{bind} method accepts regular functions of type \texttt{UserDocument}$~\rightarrow~$\texttt{User}. This is for convenience. A separate method \texttt{linearBind} could be added that accepts a functions of type
\texttt{Lin<UserDocument>}$~\rightarrow~$\texttt{Lin<User>}, or better, functions of type \texttt{Lin<UserDocument>}$~\multimap~$\texttt{Lin<User>}.


\section{Linear Embedded Language}
One can describe the language of the embedded linear types as follows:
\begin{align*}
\T &::= T \mid \cdots \mid \A \\
\A &::= \mathtt{LUnit} \mid \mathtt{LZero} \mid \mbox{\texttt{LPair<}}\T, \T\mbox{\texttt{>}} \mid \mbox{\texttt{LFun<}}\T, \T\mbox{\texttt{>}} \mid \mbox{\texttt{Lin<}}\T\mbox{\texttt{>}} \mid \mbox{\texttt{Single<}}\T\mbox{\texttt{>}}.
\end{align*}

In the embedded type system, linear types are also considered ``regular'' types and are treated the same way. An additional type-checking and/or runtime guards are required to ensure their linearity. Also, for convenience, we embed linearity inside types such as \texttt{Single}, \texttt{LFun}, and \texttt{LPair}. In other words, 
\begin{itemize}
    \item \texttt{LZero} represents $0$ and is equal to \texttt{Lin<void>},
    \item \texttt{LUnit} represents $1$ and is equal to \texttt{Lin<undefined>},
    \item \texttt{LPair<}$\T_1, \T_2$\texttt{>} represents $(\floor{\T_1}, \floor{\T_2})$,
    \item \texttt{LFun<}$\T_1, \T_2$\texttt{>} represents $\floor{\T_1} \multimap \floor{\T_2}$,
    \item \texttt{Lin<}$\T$\texttt{>} represents $\floor{\T}$, and
    \item \texttt{Single<}$\T$\texttt{>} represents $\diamond\floor{\T}$.
\end{itemize}

Linearity of \texttt{Lin} can be enforced in the runtime, simply by setting a flag or manually freeing up the memory it was allocating. The \texttt{LFun} and \texttt{LPair} can be realized as regular function and tuple types of TypeScript wrapped in a \texttt{Lin}, to ensure linearity of the result:
\begin{minted}{typescript}
export type LFun<T1, T2> = Lin<(_: T1) => T2>;
export type LPair<T1, T2> = Lin<[T1, T2]>;
\end{minted}