\chapter{Future Work}

This dissertation has presented the linear-time temporal type theory for event-driven programming. A Coq version of the theory was implemented consisting of the typing rules and an operational semantics. A generic store was implemented in Coq to provide a means of storing typing context and reasoning about it. Some TypeScript types were implemented to represent the linear and future types of the system. So, there are many ways that one can extend this work.

\section{Extend the Theory to Model Streams}

As discussed earlier, the diamond modality $\diamond\A$ models a single future value, rather than an stream of values. That was the reason the types \texttt{LPromise} and \texttt{Single} were introduced. But to model async stream of values such as \texttt{Observable}, one should extend the notion of diamond modality.

Just creating an stream out of the diamond modalities won't be sufficient as the elements are ordered:
\[
\diamond\A_0 \otimes \diamond\A_1 \otimes \diamond\A_2 \otimes \cdots.
\]
Here, $\diamond\A_0$ should resolve before $\diamond\A_1$, $\diamond\A_1$ before $\diamond\A_2$, and so on.

For the sake of example, let's assume the stream consists of two elements. One can type the stream as follows:
\[
\diamond\A_0 \otimes (\A_0 \multimap \diamond\A_1).
\]
This reads ``$\A_1$ becomes available eventually after $\A_0$ became available''.

Another approach would be using the binary operators of linear-time temporal logic, e.g. release (see \cite{DBLP:conf/focs/Pnueli77}). Let's denote release by $\clubsuit$, then $\psi~\clubsuit~\varphi$ reads ``$\varphi$ has to be true until and including the point where $\psi$ first becomes true''. So $\alpha_0~\clubsuit~\neg\alpha_1$ means ``$\alpha_1$ can resolve only after $\alpha_0$ is resolved''.

\section{Add Compile-Time Typing}

TypeScript transpiles into JavaScript, hence the type information are lost in the runtime. The new TypeScript runtime, Deno, might change that, but it's still too soon to bet. Also, TypeScript has no plan of supporting linear/affine types (see \cite{typescript-affine}).

Still, one can access the type data in the compile time. The type system can then use the already available types of \texttt{Lin} and \texttt{Single}/\texttt{LPromise} and treat them differently. ESLint can be used for this purpose, as it has access to the type information on the compile time.

\section{Generalize Preservation for Linear Types}

The preservation lemmas for $\t$ and $\e$ are introduced and proved in Coq. Although the $\t$-Preservation (Lemma \ref{lemma:t-preservation}) has full power, the $\e$-Preservation (Lemma \ref{lemma:e-preservation-weak}) was stated and proved in a weak version, assuming linear context to be empty. A stronger result can be achieved by generalizing the lemma for non-empty linear context.