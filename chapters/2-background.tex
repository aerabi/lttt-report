\chapter{Background}\label{chap:background}

Propositional logic is about judgments of the form ``$\alpha$ holds``. In contrast, a type system is about judgments of the form ``$\t$ is of type $\T$''. Although the judgments are different, to derive them one should employ similar techniques. A good example is depicted in Table \ref{tab:natural-deduction-lambda-calculus-correspondence}.

\begin{table}[!ht]
    \centering
    \begin{tabular}{c|c}
        Intuitionistic Logic & Lambda Calculus \\
        \hline \\
        $\infer[\mbox{Ax}]{\Gamma_1, \alpha, \Gamma_2 \vdash \alpha}{}$ & $\infer[]{\Gamma_1, x : \alpha, \Gamma_2 \vdash x : \alpha}{}$ \\ \\
        $\infer[\rightarrow\mbox{-I}]{\Gamma \vdash \alpha \rightarrow \beta}{\Gamma, \alpha \vdash \beta}$ & $\infer[\rightarrow\mbox{-I}]{\Gamma \vdash \lambda x. t : \alpha \rightarrow \beta}{\Gamma, x : \alpha \vdash t : \beta}$ \\ \\
        $\infer[\rightarrow\mbox{-E}]{\Gamma \vdash \beta}{\Gamma \vdash \alpha & \Gamma \vdash \alpha \rightarrow \beta}$ & $\infer[\rightarrow\mbox{-E}]{\Gamma \vdash f x : \beta}{\Gamma \vdash x : \alpha & \Gamma \vdash f : \alpha \rightarrow \beta}$
    \end{tabular}
    \caption{Correspondence between 
intuitionistic implicational natural deduction and lambda calculus type assignment rules, as stated by W.~H.~\cite{Howard1969TheFN}}
    \label{tab:natural-deduction-lambda-calculus-correspondence}
\end{table}

As one can see from Table \ref{tab:natural-deduction-lambda-calculus-correspondence}, logic and type theory are closely related. We will investigate their relation thoroughly later in this chapter. To do so, we need to define a few things first.

\section{Intuitionistic Logic}

The \textit{intuitionistic logic} is known to differ from the classical logic by rejecting the \textit{law of excluded middle} (Latin: \textit{principium tertii exclusi}, aka \textit{tertium non datur}). As a result, the two following classical tautologies are not valid in the intuitionistic logic: $p \vee \neg p$ (\textit{tertium non datur}) and $\neg \neg p \rightarrow p$ (double-negation elimination).

\subsection{Language of Propositional Intuitionistic Logic}

The language of intuitionistic propositional logic, IPC (C for calculus) or IPC$(\bot, \rightarrow, \vee, \wedge)$, is defined with the following grammar:
\[
\alpha ::= \bot \mid x \mid \alpha \rightarrow \alpha \mid \alpha \vee \alpha \mid \alpha \wedge \alpha.
\]
Here $\bot$ is constant (falsity or contradiction), $x$ ranges over the propositional variables, and the connectives are: implication $\rightarrow$, disjunction $\vee$, and conjunction $\wedge$. Negation $\neg \alpha$ is defined by $\neg \alpha = \alpha \rightarrow \bot$.

\subsection{Sequent and Natural Deduction}

A \textit{context} is a set of $\alpha$'s, usually denoted by $\Gamma$. We write $\Gamma, \Delta$ for $\Gamma \cup \Delta$, and also $\Gamma, \alpha$ for $\Gamma \cup \{\alpha\}$. 

\begin{definition}[Sequent]
The relation $\Gamma \vdash \alpha$ is defined syntactically by Figure \ref{fig:intuitionistic-logic-natural-deduction}, but can be though of as ``$\alpha$ holds if all $\Gamma$ holds''. The relation $\vdash$ is called \textit{sequent}, the elements of $\Gamma$ are called \textit{antecedents} or \textit{hypotheses}, and $\alpha$ is called \textit{succedents} or \textit{consequents}. When $\Gamma = \emptyset$, we write $\vdash \alpha$, and say that $\alpha$ is a \textit{theorem}.
\end{definition}

\begin{figure}
    \centering
    \begin{gather*}
\infer[\mbox{Var}]{\Gamma, \alpha \vdash \alpha}{}
~~~~ ~~~~
\infer[\bot\mbox{-E}]{\Gamma \vdash \beta}{\Gamma \vdash \bot}
\\ \\ 
\infer[\wedge\mbox{-I}]{\Gamma \vdash \alpha_1 \wedge \alpha_2}{
    \Gamma \vdash \alpha_1
    &
    \Gamma \vdash \alpha_2
}
~~~~ ~~~~
\infer[\wedge\mbox{-E}]{\Gamma \vdash \alpha_i}{
    \Gamma \vdash \alpha_1 \wedge \alpha_2
}
\\ \\ 
\infer[\vee\mbox{-I}]{\Gamma\vdash \alpha_1 \vee \alpha_2}{\Gamma \vdash \alpha_i}
~~~~ ~~~~
\infer[\vee\mbox{-E}]{\Gamma \vdash \beta}{
    \Gamma \vdash \alpha_1 \vee \alpha_2
    &
    \Gamma, \alpha_1 \vdash \beta
    &
    \Gamma, \alpha_2 \vdash \beta
}
\\ \\ 
\infer[\rightarrow\mbox{-I}]{\Gamma \vdash \alpha \rightarrow \beta}{\Gamma, \alpha \vdash \beta}
~~~~ ~~~~
\infer[\rightarrow\mbox{-E}]{\Gamma \vdash \beta}{\Gamma \vdash \alpha & \Gamma \vdash \alpha \rightarrow \beta}
\end{gather*}
    \caption{Natural deduction rules of the propositional intuitionistic logic}
    \label{fig:intuitionistic-logic-natural-deduction}
\end{figure}

The proof system which consists of the sequent relation together with the deduction rules, is called \textit{natural deduction}.

The subset of IPC that only deals with $\alpha ::= x \mid \alpha \rightarrow \alpha$, with three rules of Ax, $\rightarrow$-I, and -E, is called IPC$(\rightarrow)$.
\section{Simply Typed Lambda Calculus}

The language of simply typed lambda calculus $\lambda(\rightarrow)$ is defined by the following grammar:
\begin{align*}
    &\T ::= T \mid \T \rightarrow \T, \\
    &\t ::= x \mid \lambda x. \t \mid \t_1 \t_2,
\end{align*}
where $T$ designates constant types (take $\mathtt{Boolean} \mid \mathtt{String}$ as example) and $x$ designates variables. The $\t$'s are called \textit{terms} and the $\T$'s are called \textit{types}.

\subsection{Sequent and Natural Deduction}

A \textit{context} here is a sequence of ordered pairs $\ang{\t, \T}$, usually denoted by $\Gamma$. We additionally assume that there are no duplicate keys. The set of keys in $\Gamma$ is denoted by $\iota_1[\Gamma]$ and the values $\iota_2[\Gamma]$. When $\iota_1[\Gamma] \cap \iota_1[\Delta] = \emptyset$, we write $\Gamma, \Delta$ for $\Gamma \cup \Delta$. We also write $\Gamma, \t: \T$ for $\Gamma, \{\ang{\t, \T}\}$.

\begin{definition}
The relation $\Gamma \vdash \t : \T$ for $\lambda(\rightarrow)$ is defined syntactically by \figref{fig:simply-typed-lambda-calculus-natural-deduction}.
\end{definition}

\begin{figure}
    \centering
    \begin{gather*}
\infer[\mbox{Var}]{\Gamma, x : \T \vdash x : \T}{}
\\ \\ 
\infer[\rightarrow\mbox{-I}]{\Gamma \vdash \lambda x. \t : \T \rightarrow \mathfrak{S}}{\Gamma, x : \T \vdash \t : \mathfrak{S}}
~~~~ ~~~~
\infer[\rightarrow\mbox{-E}]{\Gamma \vdash \t_1\t_2 : \mathfrak{S}}{
    \Gamma \vdash \t_1 : \T \rightarrow \mathfrak{S}
    &
    \Gamma \vdash \t_2 : \T
}
\end{gather*}
    \caption{Natural deduction rules of the simply typed lambda calculus}
    \label{fig:simply-typed-lambda-calculus-natural-deduction}
\end{figure}

\section{Curry--Howard Correspondence}

The following theorem, states the Curry--Howard correspondence between implicational fragment of intuitionistic propositional logic IPC$(\rightarrow)$ and the simply typed lambda calculus $\lambda(\rightarrow)$.

\begin{theorem}
The types in $\lambda(\rightarrow)$ form an \emph{IPC}$(\rightarrow)$. Furthermore:
\begin{itemize}
    \item If $\Gamma \vdash \t : \alpha$, then $\iota_2[\Gamma] \vdash \alpha$.
    \item If $\Delta \vdash \alpha$, then there is a $\Gamma$ and a $\t$, such that $\Delta = \iota_2[\Gamma]$ and $\Gamma \vdash \t : \alpha$.
\end{itemize}
\end{theorem}

An interesting case is when $\Gamma = \emptyset$.

\begin{corollary}
\label{corollary:theorem-if-inhabited}
A formula $\varphi$ in \emph{IPC}$(\rightarrow)$ is a true theorem, iff it's inhabited by a $\lambda(\rightarrow)$-term $\t$, that is $\vdash \t : \varphi$.
\end{corollary}

An example is the formula $(\beta \rightarrow \alpha) \rightarrow (\gamma \rightarrow \beta) \rightarrow \gamma \rightarrow \alpha$. Only the fact that the $\lambda(\rightarrow)$-term $\lambda a. \lambda b. \lambda c. a (b c)$ inhabits this type, proves it. \figref{fig:curry-howard-example} demonstrate a proof of it by natural deduction. For neatness, we named $\Gamma := a : \beta \rightarrow \alpha, b : \gamma \rightarrow \beta, c : \gamma$.

\begin{figure}
    \centering
    \begin{equation*}
\infer{\vdash \lambda a. \lambda b. \lambda c. a (b c) : (\beta \rightarrow \alpha) \rightarrow (\gamma \rightarrow \beta) \rightarrow \gamma \rightarrow \alpha}{
%
\infer{a : \beta \rightarrow \alpha \vdash \lambda b. \lambda c. a (b c) : (\gamma \rightarrow \beta) \rightarrow \gamma \rightarrow \alpha}{
%
\infer{a : \beta \rightarrow \alpha, b : \gamma \rightarrow \beta \vdash \lambda c. a (b c) : \gamma \rightarrow \alpha}{
%
\infer{a : \beta \rightarrow \alpha, b : \gamma \rightarrow \beta, c : \gamma \vdash a (b c) : \alpha}{
%
\infer{\Gamma \vdash a : \beta \rightarrow \alpha}{}
&
\infer{\Gamma \vdash b c : \beta}{
%
\infer{\Gamma \vdash b : \gamma \rightarrow \beta}{}
&
\infer{\Gamma \vdash c : \gamma}{}
}
}
}
}
}
\end{equation*}
    \caption{Proof by natural deduction that $\vdash \lambda a. \lambda b. \lambda c. a (b c) : (\beta \rightarrow \alpha) \rightarrow (\gamma \rightarrow \beta) \rightarrow \gamma \rightarrow \alpha$}
    \label{fig:curry-howard-example}
\end{figure}

A case of Corollary \ref{corollary:theorem-if-inhabited} is $\varphi = \bot$. If any term of the language inhabits the contradiction, then the underlying logic on the types is inconsistent.
\section{Structural Rules}

In natural deduction, \textit{structural rules} are inference rules that do not refer to any logical connective, but instead operates on the sequents directly.

Three common structural rules are:
\begin{itemize}
    \item \textbf{Weakening}. Which states that hypotheses of a sequent could be extended.
    \begin{gather*}
        \infer=[\mbox{Weak}]{\Gamma_1, \alpha, \Gamma_2 \vdash \beta}{\Gamma_1, \Gamma_2 \vdash \beta}
        ~~~~ ~~~~
        \infer=[\mbox{Weak}]{\Gamma_1, x : \A, \Gamma_2\vdash \t : \mathfrak{B}}{\Gamma_1, \Gamma_2\vdash \t : \mathfrak{B}}
    \end{gather*}
    \item \textbf{Contraction}. Which states that two similar (or unifiable) hypothesis can be replaced by a single hypothesis.
    \begin{gather*}
        \infer=[\mbox{Contr}]{\Gamma_1, \alpha, \Gamma_2 \vdash \beta}{\Gamma_1, \alpha, \alpha, \Gamma_2 \vdash \beta}
        ~~~~ ~~~~
        \infer=[\mbox{Contr}]{\Gamma_1, x : \A, \Gamma_2 \vdash \t : \mathfrak{B}}{\Gamma_1, x : \A, x : \A, \Gamma_2 \vdash \t : \mathfrak{B}}
    \end{gather*}
    Another version of the weakening rule is proposed by \cite{DavidWalker2004} for type systems:
    \begin{gather*}
        \infer=[\mbox{Contr}]{\Gamma_1, x_1 : \A, \Gamma_2 \vdash \t[x_2 \mapsto x_1, x_3 \mapsto x_1] : \mathfrak{B}}{\Gamma_1, x_2 : \A, x_3 : \A, \Gamma_2 \vdash \t : \mathfrak{B}}
    \end{gather*}
    \item \textbf{Exchange}. Which states that two hypotheses may be swapped.
    \begin{gather*}
        \infer=[\mbox{Exch}]{\Gamma_1, \alpha_2, \alpha_1, \Gamma_2 \vdash \beta}{\Gamma_1, \alpha_1, \alpha_2, \Gamma_2 \vdash \beta}
        ~~~~ ~~~~
        \infer=[\mbox{Exch}]{\Gamma_1, x_2 : \A_2, x_1 : \A_1 \Gamma_2 \vdash \t : \mathfrak{B}}{\Gamma_1, x_1 : \A_1, x_2 : \A_2, \Gamma_2 \vdash \t : \mathfrak{B}}
    \end{gather*}
\end{itemize}

Any logic (or type system) that lacks any of these structural rules is called a substructural logic (or type system, respectively).

\begin{table}[!ht]
    \centering
    \begin{tabular}{l|ccc}
         & W & C & E \\
        \hline
        Unrestricted & \checkmark & \checkmark & \checkmark \\
        Affine & \checkmark &  & \checkmark \\
        Relevant &  & \checkmark & \checkmark \\
        Linear & & & \checkmark \\
        Ordered
    \end{tabular}
    \caption{Substructural logics / type systems and the structural rules they have}
    \label{tab:my_label}
\end{table}

To understand how the structural rules for the type systems affect programming languages they are used for, a list of substructural type systems is presented with the restrictions they have:
\begin{itemize}
    \item \textbf{Affine}. Every variable can be used at most once.
    \item \textbf{Relevant}. Every variable must be used at least once.
    \item \textbf{Linear}. Every variable must be used exactly once.
    \item \textbf{Ordered}. Every variable must be used exactly once and in the order in which it is introduced.
\end{itemize}

\section{Intuitionistic Logic with Structural Rules}

\begin{figure}
    \centering
    \begin{gather*}
\infer[\mbox{Var}]{\Gamma, \alpha \vdash \alpha}{}
~~~~ ~~~~
\infer[\bot\mbox{-E}]{\Gamma \vdash \beta}{\Gamma \vdash \bot}
\\ \\ 
\infer[\wedge\mbox{-I}]{\Gamma \vdash \alpha_1 \wedge \alpha_2}{
    \Gamma \vdash \alpha_1
    &
    \Gamma \vdash \alpha_2
}
~~~~ ~~~~
\infer[\wedge\mbox{-E}]{\Gamma \vdash \alpha_i}{
    \Gamma \vdash \alpha_1 \wedge \alpha_2
}
\\ \\ 
\infer[\vee\mbox{-I}]{\Gamma\vdash \alpha_1 \vee \alpha_2}{\Gamma \vdash \alpha_i}
~~~~ ~~~~
\infer[\vee\mbox{-E}]{\Gamma \vdash \beta}{
    \Gamma \vdash \alpha_1 \vee \alpha_2
    &
    \Gamma, \alpha_1 \vdash \beta
    &
    \Gamma, \alpha_2 \vdash \beta
}
\\ \\ 
\infer[\rightarrow\mbox{-I}]{\Gamma \vdash \alpha \rightarrow \beta}{\Gamma, \alpha \vdash \beta}
~~~~ ~~~~
\infer[\rightarrow\mbox{-E}]{\Gamma \vdash \beta}{\Gamma \vdash \alpha & \Gamma \vdash \alpha \rightarrow \beta}
\end{gather*}
    \caption{Natural deduction rules of the propositional intuitionistic logic, assuming contraction}
    \label{fig:intuitionistic-logic-natural-deduction-contraction}
\end{figure}

Depending on the structural rules one assumes, the inference rules may differ. Usually, contraction is assumed for the intuitionistic logic, so the inference rules are represented as in \figref{fig:intuitionistic-logic-natural-deduction-contraction}.

For example, using the $\rightarrow$-E rule from \figref{fig:intuitionistic-logic-natural-deduction}, there is a contraction happening under the hood:
\[
\infer={\Gamma \vdash \beta}{
    \infer[\mbox{$\rightarrow$-E}]{\Gamma, \Gamma \vdash \beta}{
        \Gamma \vdash \alpha
        &
        \Gamma \vdash \alpha \rightarrow \beta
    }
}
\]

Similarly, if one does not assume exchange, the axiom will become $\Gamma_1, \alpha, \Gamma_2 \vdash \alpha$ instead of simply $\Gamma, \alpha \vdash \alpha$.

From now on, we assume exchange for all of the inference rules, and contraction for the non-linear logics and type systems.
\section{A Taste of Linear Logic}

Linear type systems correspond to or are inspired by the linear logic of \cite{DBLP:journals/tcs/Girard87}. In ordinary logic, we think of $\psi \vdash \varphi$ as ``$\varphi$ is true whenever $\psi$ is'', and when the truth of something is proven, it is true for the eternity. That is not the case with the linear logic. Members of the logic are not facts about the nature, rather resources that are consumed.

A good example is the following. In ordinary logic (take intuitionistic for example), one can prove
\[
\alpha \rightarrow \beta, \alpha \rightarrow \gamma, \alpha \vdash \beta \wedge \gamma.
\]
But this is not the case for the linear logic:
\[
\alpha \multimap \beta, \alpha \multimap \gamma, \alpha \not\vdash \beta \otimes \gamma.
\]
Once the hypothesis $\alpha$ is used to derive $\beta$, it's not available any more. We need another $\alpha$ to derive $\gamma$:
\[
\alpha \multimap \beta, \alpha \multimap \gamma, \alpha, \alpha \vdash \beta \otimes \gamma.
\]

As \cite{DBLP:conf/mfcs/Wadler93} puts it: ``Traditional logic encourages reckless use of resources. Contraction profligately duplicates assumptions, Weakening foolishly discards assumptions. This makes sense for logic, where truth is free; and it makes sense for some programming languages, where copying a value is as cheap as copying a pointer. But it is not always sensible.''

The language of linear logic, LL$(0, \multimap, \oplus, \otimes, !)$ or simply LL, is defined with the following grammar:
\[
\alpha ::= 0 \mid \alpha \multimap \alpha \mid \alpha \oplus \alpha \mid \alpha \otimes \alpha \mid~ !\alpha.
\]
Here $0$ is a constant and the linear version of $\bot$, the modality $!$ is called \textit{of course} or \textit{bang} and is similar to the modal $\square$, the connective $\multimap$ is called \textit{lollipop} which is the bilinear version of implication, $\oplus$ is called \textit{additive disjunction} is the bilinear version of or, and $\otimes$ is called \textit{multiplicative conjunction} or \textit{tensor} is the bilinear version of and.

The linear logic of \cite{DBLP:journals/tcs/Girard87} is more extensive and has a richer inventory of connectives and modalities. An honorable mention is the modality $?$, that is called \textit{why not} or \textit{par}, and is the DeMorgan dual of $!$. As you have already guessed, $?$ is the linear version of $\diamond$. The two modalities $!$ and $?$ are called \textit{exponential} modalities.

\section{Linear/Non-Linear Logic}

It is usually not enough to have only linear members in the logic or type system. In many domains, when using the linear types, they are mixed with unrestricted ones. As an example, \cite{DavidWalker2004} prefixes his types with \texttt{lin} or \texttt{un} and treats them differently.

In the linear logic, the modality $!$ is used for addressing this shortcoming. Members of the form $!\alpha$ can be duplicated or discarded. But it's not practical to be used in a type system, as it has very limited flexibility.
%
The linear/non-linear logic of \cite{DBLP:conf/csl/Benton94} offers a more practical modeling of linear and non-linear logics mixed together.

The language of the linear/non-linear logic, LNL, is defined with the following grammar:
\begin{align*}
    &\tau ::= \top \mid t \mid \tau \times \tau \mid \tau + \tau \mid \tau \rightarrow \tau \mid \ceil{\alpha}, \\
    &\alpha ::= 1 \mid a \mid \alpha \otimes \alpha \mid \alpha \oplus \alpha \mid \alpha \multimap \alpha \mid \floor{\tau}.
\end{align*}
Here $a$ (resp. $t$) ranges over a set of linear (resp. non-linear) atomic propositions / types, the functor $\ceil{\cdot}$ is called \textit{lift}, and the functor $\floor{\cdot}$ is called \textit{lower}. These two functors are in charge of moving between the linear and the non-linear worlds. The two connectives $\times$ and $+$ denote conjunction and disjunction.

\section{Modal Logic}

Modal logic is a collection of formal systems initially created to reason about necessity and possibility. These two are denoted by two modalities $\diamond\varphi$, that reads ``possibly $\varphi$'', and $\square\varphi$, that reads ``necessarily $\varphi$''.

So, the language of the modal logic is defined by the following grammar:
\[
\alpha ::= \bot \mid a \mid \alpha \rightarrow \alpha \mid \alpha \vee \alpha \mid \alpha \wedge \alpha \mid \diamond\alpha \mid \square\alpha.
\]
Here the modalities $\diamond$ and $\square$ are De Morgan duals:
\begin{itemize}
    \item $\diamond\varphi$ ``possibly $\varphi$'' is equivalent to $\neg\square\neg\varphi$ ``not necessarily not $\varphi$'',
    \item $\square\varphi$ ``necessarily $\varphi$'' is equivalent to $\neg\diamond\neg\varphi$ ``not possibly not $\varphi$''.
\end{itemize}

The following are a few well-known axioms in modal logic:
\begin{itemize}
    \item \textbf{N}. $\vdash \varphi$ then $\vdash \square\varphi$.
    \item \textbf{K}. $\square(\varphi \rightarrow \psi) \rightarrow (\square\varphi \rightarrow \square\psi)$.
    \item \textbf{T}. $\square\varphi \rightarrow \varphi$.
    \item \textbf{4}. $\square\varphi \rightarrow \square\square\varphi$.
    \item \textbf{5}. $\diamond\varphi \rightarrow \square\diamond\varphi$.
    \item \textbf{B}. $\varphi \rightarrow \square\diamond\varphi$.
    \item \textbf{D}. $\square\varphi \rightarrow \diamond\varphi$.
\end{itemize}
A modal logic may adopt some of these axioms and reject some. E.g. S5, one of the well-known modal logics, adopts either all except \textbf{5} and \textbf{D}, or all except \textbf{4}, \textbf{B}, and \textbf{D}.
\section{Temporal Logic}

Temporal logic as an umbrella term is the study of reasoning about propositions qualified in terms of time. Formulating temporal logic as a modal logic was first done by \cite{PRIOR1957,PRIOR1967,prior1968papers} and then ported to computer science by \cite{DBLP:conf/focs/Pnueli77}.

Prior's 4 tense operators were the following (see \cite{sep-logic-temporal}):
\begin{itemize}
    \item $P\varphi$, ``It has at some time been the case that $\varphi$''.
    \item $F\varphi$, ``It will at some time be the case that $\varphi$''.
    \item $H\varphi$, ``It has always been the case that $\varphi$''.
    \item $G\varphi$, ``It will always be the case that $\varphi$''.
\end{itemize}

As we are not interested in the past, we will omit $P$ and $H$, and show the other two, $F$ and $G$, with $\diamond$ and $\square$, respectively. Note that $F$ and $G$ form a modal logic, and e.g. $\neg{}F\neg\varphi$ ``not eventually not $\varphi$'' is equivalent to $G\varphi$ ``always $\varphi$''. Here $G$ is always in the sense of ``from now on''.

The language of temporal logic is then the same as the language of modal logic, but one can add more operators:
\begin{itemize}
    \item Next. $N\varphi$, or $\circ\varphi$, ``$\varphi$ holds in the next stage''.
    \item Until. $\varphi~U~\psi$, or $\varphi~\spadesuit~\psi$, ``$\varphi$ (holds at least) until $\psi$ (and $\psi$ has to hold eventually)''.
    \item Release. $\varphi~R~\psi$, or $\varphi~\clubsuit~\psi$, ``$\varphi$ releases $\psi$'', or ``$\psi$ is true until and including the point where $\varphi$ becomes true (if $\varphi$ never becomes true, $\psi$ has to stay true forever)''.
\end{itemize}

\section{Linear-Time Temporal Logic}

Linear-time temporal logic (LTL) is a modal temporal logic reasoning about the linear time, in contrast to the branching-time temporal logics such as computation tree logic (CTL). LTL was introduced by \cite{DBLP:conf/focs/Pnueli77}.

LTL adds two more operators to the temporal logic:
\begin{itemize}
    \item Weak until. $\varphi~W~\psi$, or $\varphi~\varheartsuit~\psi$, ``$\varphi$ (holds at least) until $\psi$ (and $\varphi$ holds forever if $\psi$ never becomes true)''.
    \item Strong release. $\varphi~M~\psi$, or $\varphi~\vardiamondsuit~\psi$, ``$\psi$ is true until and including the point where $\varphi$ becomes true (and $\varphi$ has to hold eventually)''.
\end{itemize}

The following equivalences hold in LTL:
\begin{itemize}
    \item Next is self-dual: $\neg\circ\neg\varphi \equiv \circ\varphi$.
    \item Future/eventually and always are dual:
    \begin{eqnarray*}
    \neg\diamond\neg\varphi &\equiv& \square\varphi, \\
    \neg\square\neg\varphi &\equiv& \diamond\varphi.
    \end{eqnarray*}
    \item Until and release are dual: 
    \begin{eqnarray*}
    \neg(\neg\varphi~\spadesuit~\neg\psi) &\equiv& \varphi~\clubsuit~\psi, \\
    \neg(\neg\varphi~\clubsuit~\neg\psi) &\equiv& \varphi~\spadesuit~\psi.
    \end{eqnarray*}
    \item Weak until and strong release are dual: 
    \begin{eqnarray*}
    \neg(\neg\varphi~\varheartsuit~\neg\psi) &\equiv& \varphi~\vardiamondsuit~\psi, \\
    \neg(\neg\varphi~\vardiamondsuit~\neg\psi) &\equiv& \varphi~\varheartsuit~\psi.
    \end{eqnarray*}
\end{itemize}

Some special temporal axioms in LTL are the following:
\begin{itemize}
    \item $\square\varphi \rightarrow \square\square\varphi$ (4).
    \item $\square\square\varphi \rightarrow \square\varphi$ (follows from T).
    \item $\diamond\varphi \rightarrow \diamond\diamond\varphi$ (follows from 5 and D). 
    \item $\diamond\diamond\varphi \rightarrow \diamond\varphi$.
\end{itemize}

% \begin{theorem}
% For formulae $\varphi$ and $\psi$ we have: $\diamond\varphi \wedge (\varphi \rightarrow \diamond\psi) \rightarrow \diamond\psi$.
% \end{theorem}
% \begin{proof}
% \end{proof}