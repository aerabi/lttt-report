\chapter{Background}\label{chap:background}

Propositional logic is about judgments of the form ``$\alpha$ holds``. In contrast, a type system is about judgments of the form ``$\t$ is of type $\T$''. Although the judgments are different, to derive them one should employ similar techniques. A good example is depicted in Table \ref{tab:natural-deduction-lambda-calculus-correspondence}.

\begin{table}[!ht]
    \centering
    \begin{tabular}{c|c}
        Intuitionistic Logic & Lambda Calculus \\
        \hline \\
        $\infer[\mbox{Ax}]{\Gamma_1, \alpha, \Gamma_2 \vdash \alpha}{}$ & $\infer[]{\Gamma_1, x : \alpha, \Gamma_2 \vdash x : \alpha}{}$ \\ \\
        $\infer[\rightarrow\mbox{-I}]{\Gamma \vdash \alpha \rightarrow \beta}{\Gamma, \alpha \vdash \beta}$ & $\infer[\rightarrow\mbox{-I}]{\Gamma \vdash \lambda x. t : \alpha \rightarrow \beta}{\Gamma, x : \alpha \vdash t : \beta}$ \\ \\
        $\infer[\rightarrow\mbox{-E}]{\Gamma \vdash \beta}{\Gamma \vdash \alpha & \Gamma \vdash \alpha \rightarrow \beta}$ & $\infer[\rightarrow\mbox{-E}]{\Gamma \vdash f x : \beta}{\Gamma \vdash x : \alpha & \Gamma \vdash f : \alpha \rightarrow \beta}$
    \end{tabular}
    \caption{Correspondence between 
intuitionistic implicational natural deduction and lambda calculus type assignment rules by W.~H.~\cite{Howard1969TheFN}}
    \label{tab:natural-deduction-lambda-calculus-correspondence}
\end{table}

As one can see from Table \ref{tab:natural-deduction-lambda-calculus-correspondence}, logic and type theory are closely related. We will investigate their relation thoroughly later in this chapter, but we need to define a few things first.

\section{Intuitionistic Logic}

The \textit{intuitionistic logic} is known to differ from the classical logic by rejecting the \textit{law of excluded middle} (Latin: \textit{principium tertii exclusi}, aka \textit{tertium non datur}). As a result, the two following classical tautologies are not valid in the intuitionistic logic: $p \vee \neg p$ (\textit{tertium non datur}) and $\neg \neg p \rightarrow p$ (double-negation elimination).

\subsection{Language of Propositional Intuitionistic Logic}

The language of intuitionistic propositional logic, IPC (C for calculus) or IPC$(\bot, \rightarrow, \vee, \wedge)$, is defined with the following grammar:
\[
\alpha ::= \bot \mid x \mid \alpha \rightarrow \alpha \mid \alpha \vee \alpha \mid \alpha \wedge \alpha.
\]
Here $\bot$ is constant (falsity or contradiction), $x$ ranges over the propositional variables, and the connectives are: implication $\rightarrow$, disjunction $\vee$, and conjunction $\wedge$. Negation $\neg \alpha$ is defined by $\neg \alpha = \alpha \rightarrow \bot$.

\subsection{Sequent and Natural Deduction}

A \textit{context} is a set of $\alpha$'s, usually denoted by $\Gamma$. We write $\Gamma, \Delta$ for $\Gamma \cup \Delta$, and also $\Gamma, \alpha$ for $\Gamma \cup \{\alpha\}$. 

\begin{definition}[Sequent]
The relation $\Gamma \vdash \alpha$ is defined syntactically by Figure \ref{fig:intuitionistic-logic-natural-deduction}, but can be though of as ``$\alpha$ holds if all $\Gamma$ holds''. The relation $\vdash$ is called \textit{sequent}, the elements of $\Gamma$ are called \textit{antecedents} or \textit{hypotheses}, and $\alpha$ is called \textit{succedents} or \textit{consequents}. When $\Gamma = \emptyset$, we write $\vdash \alpha$, and say that $\alpha$ is a \textit{theorem}.
\end{definition}

\begin{figure}
    \centering
    \begin{gather*}
\infer[\mbox{Var}]{\Gamma, \alpha \vdash \alpha}{}
~~~~ ~~~~
\infer[\bot\mbox{-E}]{\Gamma \vdash \beta}{\Gamma \vdash \bot}
\\ \\ 
\infer[\wedge\mbox{-I}]{\Gamma \vdash \alpha_1 \wedge \alpha_2}{
    \Gamma \vdash \alpha_1
    &
    \Gamma \vdash \alpha_2
}
~~~~ ~~~~
\infer[\wedge\mbox{-E}]{\Gamma \vdash \alpha_i}{
    \Gamma \vdash \alpha_1 \wedge \alpha_2
}
\\ \\ 
\infer[\vee\mbox{-I}]{\Gamma\vdash \alpha_1 \vee \alpha_2}{\Gamma \vdash \alpha_i}
~~~~ ~~~~
\infer[\vee\mbox{-E}]{\Gamma \vdash \beta}{
    \Gamma \vdash \alpha_1 \vee \alpha_2
    &
    \Gamma, \alpha_1 \vdash \beta
    &
    \Gamma, \alpha_2 \vdash \beta
}
\\ \\ 
\infer[\rightarrow\mbox{-I}]{\Gamma \vdash \alpha \rightarrow \beta}{\Gamma, \alpha \vdash \beta}
~~~~ ~~~~
\infer[\rightarrow\mbox{-E}]{\Gamma \vdash \beta}{\Gamma \vdash \alpha & \Gamma \vdash \alpha \rightarrow \beta}
\end{gather*}
    \caption{Natural deduction rules of the propositional intuitionistic logic}
    \label{fig:intuitionistic-logic-natural-deduction}
\end{figure}

The proof system which consists of the sequent relation together with the deduction rules, is called \textit{natural deduction}.

The subset of IPC that only deals with $\alpha ::= x \mid \alpha \rightarrow \alpha$, with three rules of Ax, $\rightarrow$-I, and -E, is called IPC$(\rightarrow)$.
\section{Simply Typed Lambda Calculus}

The language of simply typed lambda calculus $\lambda(\rightarrow)$ is defined by the following grammar:
\begin{align*}
    &\T ::= T \mid \T \rightarrow \T, \\
    &\t ::= x \mid \lambda x. \t \mid \t_1 \t_2,
\end{align*}
where $T$ designates constant types (take $\mathtt{Boolean} \mid \mathtt{String}$ as example) and $x$ designates variables. The $\t$'s are called \textit{terms} and the $\T$'s are called \textit{types}.

\subsection{Sequent and Natural Deduction}

A \textit{context} here is a sequence of ordered pairs $\ang{\t, \T}$, usually denoted by $\Gamma$. We additionally assume that there are no duplicate keys. The set of keys in $\Gamma$ is denoted by $\iota_1[\Gamma]$ and the values $\iota_2[\Gamma]$. When $\iota_1[\Gamma] \cap \iota_1[\Delta] = \emptyset$, we write $\Gamma, \Delta$ for $\Gamma \cup \Delta$. We also write $\Gamma, \t: \T$ for $\Gamma, \{\ang{\t, \T}\}$.

\begin{definition}
The relation $\Gamma \vdash \t : \T$ for $\lambda(\rightarrow)$ is defined syntactically by \figref{fig:simply-typed-lambda-calculus-natural-deduction}.
\end{definition}

\begin{figure}
    \centering
    \begin{gather*}
\infer[\mbox{Var}]{\Gamma, x : \T \vdash x : \T}{}
\\ \\ 
\infer[\rightarrow\mbox{-I}]{\Gamma \vdash \lambda x. \t : \T \rightarrow \mathfrak{S}}{\Gamma, x : \T \vdash \t : \mathfrak{S}}
~~~~ ~~~~
\infer[\rightarrow\mbox{-E}]{\Gamma \vdash \t_1\t_2 : \mathfrak{S}}{
    \Gamma \vdash \t_1 : \T \rightarrow \mathfrak{S}
    &
    \Gamma \vdash \t_2 : \T
}
\end{gather*}
    \caption{Natural deduction rules of the simply typed lambda calculus}
    \label{fig:simply-typed-lambda-calculus-natural-deduction}
\end{figure}

\section{Curry--Howard Correspondence}

The following theorem, states the Curry--Howard correspondence between implicational fragment of intuitionistic propositional logic IPC$(\rightarrow)$ and the simply typed lambda calculus $\lambda(\rightarrow)$.

\begin{theorem}
The types in $\lambda(\rightarrow)$ form an \emph{IPC}$(\rightarrow)$. Furthermore:
\begin{itemize}
    \item If $\Gamma \vdash \t : \alpha$, then $\iota_2[\Gamma] \vdash \alpha$.
    \item If $\Delta \vdash \alpha$, then there is a $\Gamma$ and a $\t$, such that $\Delta = \iota_2[\Gamma]$ and $\Gamma \vdash \t : \alpha$.
\end{itemize}
\end{theorem}

An interesting case is when $\Gamma = \emptyset$.

\begin{corollary}
\label{corollary:theorem-if-inhabited}
A formula $\varphi$ in \emph{IPC}$(\rightarrow)$ is a true theorem, iff it's inhabited by a $\lambda(\rightarrow)$-term $\t$, that is $\vdash \t : \varphi$.
\end{corollary}

An example is the formula $(\beta \rightarrow \alpha) \rightarrow (\gamma \rightarrow \beta) \rightarrow \gamma \rightarrow \alpha$. Only the fact that the $\lambda(\rightarrow)$-term $\lambda a. \lambda b. \lambda c. a (b c)$ inhabits this type, proves it. \figref{fig:curry-howard-example} demonstrate a proof of it by natural deduction. For neatness, we named $\Gamma := a : \beta \rightarrow \alpha, b : \gamma \rightarrow \beta, c : \gamma$.

\begin{figure}
    \centering
    \begin{equation*}
\infer{\vdash \lambda a. \lambda b. \lambda c. a (b c) : (\beta \rightarrow \alpha) \rightarrow (\gamma \rightarrow \beta) \rightarrow \gamma \rightarrow \alpha}{
%
\infer{a : \beta \rightarrow \alpha \vdash \lambda b. \lambda c. a (b c) : (\gamma \rightarrow \beta) \rightarrow \gamma \rightarrow \alpha}{
%
\infer{a : \beta \rightarrow \alpha, b : \gamma \rightarrow \beta \vdash \lambda c. a (b c) : \gamma \rightarrow \alpha}{
%
\infer{a : \beta \rightarrow \alpha, b : \gamma \rightarrow \beta, c : \gamma \vdash a (b c) : \alpha}{
%
\infer{\Gamma \vdash a : \beta \rightarrow \alpha}{}
&
\infer{\Gamma \vdash b c : \beta}{
%
\infer{\Gamma \vdash b : \gamma \rightarrow \beta}{}
&
\infer{\Gamma \vdash c : \gamma}{}
}
}
}
}
}
\end{equation*}
    \caption{Proof by natural deduction that $\vdash \lambda a. \lambda b. \lambda c. a (b c) : (\beta \rightarrow \alpha) \rightarrow (\gamma \rightarrow \beta) \rightarrow \gamma \rightarrow \alpha$}
    \label{fig:curry-howard-example}
\end{figure}

A case of Corollary \ref{corollary:theorem-if-inhabited} is $\varphi = \bot$. If any term of the language inhabits the contradiction, then the underlying logic on the types is inconsistent.